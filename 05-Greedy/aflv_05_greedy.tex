\documentclass[12pt,t]{beamer}
\subtitle{05 -- Greedy}
\newcommand{\shorttitle}{Greedy}
\usepackage{graphicx}
\setbeameroption{hide notes}
\setbeamertemplate{note page}[plain]

% get rid of junk
\usetheme{default}
\beamertemplatenavigationsymbolsempty
\hypersetup{pdfpagemode=UseNone} % don't show bookmarks on initial view

% font
\usepackage{fontspec}
\setsansfont{TeX Gyre Heros}
\setbeamerfont{note page}{family*=pplx,size=\footnotesize} % Palatino for notes
% "TeX Gyre Heros can be used as a replacement for Helvetica"
% In Unix, unzip the following into ~/.fonts
% In Mac, unzip it, double-click the .otf files, and install using "FontBook"
%   http://www.gust.org.pl/projects/e-foundry/tex-gyre/heros/qhv2.004otf.zip

% named colors
\definecolor{offwhite}{RGB}{249,242,215}
% \definecolor{foreground}{RGB}{255,255,255}
\definecolor{foreground}{RGB}{0,0,0}
% \definecolor{background}{RGB}{24,24,24}
\definecolor{background}{RGB}{255,255,255}
\definecolor{title}{RGB}{107,174,214}
\definecolor{gray}{RGB}{100,100,100}
\definecolor{subtitle}{RGB}{102,155,204}
\definecolor{hilight}{RGB}{20,180,204}
\definecolor{vhilight}{RGB}{255,111,207}
\definecolor{lolight}{RGB}{155,155,155}
%\definecolor{green}{RGB}{125,250,125}

% use those colors
\setbeamercolor{titlelike}{fg=title}
\setbeamercolor{subtitle}{fg=subtitle}
\setbeamercolor{institute}{fg=gray}
\setbeamercolor{normal text}{fg=foreground,bg=background}
\setbeamercolor{item}{fg=foreground} % color of bullets
\setbeamercolor{subitem}{fg=gray}
\setbeamercolor{itemize/enumerate subbody}{fg=gray}
\setbeamertemplate{itemize subitem}{{\textendash}}
\setbeamerfont{itemize/enumerate subbody}{size=\footnotesize}
\setbeamerfont{itemize/enumerate subitem}{size=\footnotesize}

% settings for table of contents
\setbeamercolor{section in toc}{fg=foreground,bg=background}
\setbeamerfont{subsection in toc}{size=\footnotesize}
\setbeamertemplate{section in toc}{{\scriptsize\leavevmode\raise1.35pt\hbox{$\blacktriangleright$}} \inserttocsection}
%\setbeamertemplate{subsection in toc}{\quad{\tiny\leavevmode\raise1.5pt\hbox{$\blacktriangleright$}} \footnotesize\inserttocsubsection\\}
\setbeamertemplate{subsection in toc}{\quad\quad\textendash\enspace\inserttocsubsection\\}

% page number
\setbeamertemplate{footline}{%
    \raisebox{5pt}{\makebox[\paperwidth]{\hfill\makebox[20pt]{\color{gray}
          \scriptsize\insertframenumber}}}\hspace*{5pt}}

% add a bit of space at the top of the notes page
\addtobeamertemplate{note page}{\setlength{\parskip}{12pt}}

% add subsection as frame title when it is empty
\makeatletter
  \CheckCommand*\beamer@checkframetitle{%
    \@ifnextchar\bgroup\beamer@inlineframetitle{}}
  \renewcommand*\beamer@checkframetitle{%
    \global\let\beamer@frametitle\relax\@ifnextchar%
    \bgroup\beamer@inlineframetitle{}}
\makeatother

\addtobeamertemplate{frametitle}{
  \ifx\insertframetitle\empty
      \frametitle{\insertsubsectionhead}
  \else
  \fi
 }{}


\usepackage{pbox}

% a few macros
\newcommand{\bi}{\begin{itemize}}
\newcommand{\ei}{\end{itemize}}
\newcommand{\ig}{\includegraphics}
\newcommand{\subt}[1]{{\footnotesize \color{subtitle} {#1}}}

% title info
\title{Kompetitives Programmieren}
%\author{Gregor Behnke}
\institute{Prof. Dr. Susanne Biundo-Stephan \\ Gregor Behnke \\ Institute of Artificial Intelligence\\ Ulm University}
\date{\tiny based on Bjarki Ágúst Guðmundsson's and Tómas Ken Magnússon's\\Competitive Programming}


\setbeamertemplate{footline}[text line]{%
  \parbox{\linewidth}{\vspace*{-8pt}\shorttitle \hfill \insertframenumber/\inserttotalframenumber}}
\setbeamertemplate{navigation symbols}{}


\newcommand{\specialcell}[2][c]{%
  \begin{tabular}[#1]{@{}c@{}}#2\end{tabular}}

% Tikz
\usepackage{tikz}
\usepackage{tkz-euclide}
\usetikzlibrary{arrows,shapes}
% \usepackage{intersections}
\usetkzobj{all}
\usetikzlibrary{arrows,shapes,angles,quotes,shapes, calc, decorations,matrix}
\usepackage{forest}
\pgfdeclarelayer{bg}    % declare background layer
\pgfsetlayers{bg,main}


% Minted
\usepackage{minted}
\usemintedstyle{tango}
\newminted{cpp}{fontsize=\footnotesize}

% Graph styles
\tikzstyle{vertex}=[circle,fill=black!50,minimum size=15pt,inner sep=0pt, font=\small]
\tikzstyle{selected vertex} = [vertex, fill=red!24]
\tikzstyle{selected2 vertex} = [vertex, fill=hilight!50, text=black]
\tikzstyle{vertex1} = [vertex, fill=red]
\tikzstyle{vertex2} = [vertex, fill=blue]
\tikzstyle{vertex3} = [vertex, fill=green, text=black]
\tikzstyle{vertex4} = [vertex, fill=yellow, text=black]
\tikzstyle{vertex5} = [vertex, fill=pink, text=black]
\tikzstyle{vertex6} = [vertex, fill=purple]
\tikzstyle{edge} = [draw,thick,-]
\tikzstyle{dedge} = [draw,thick,->]
\tikzstyle{weight} = [font=\scriptsize,pos=0.5]
\tikzstyle{selected edge} = [draw,line width=2pt,-,red!50]
\tikzstyle{ignored edge} = [draw,line width=5pt,-,black!20]


\begin{document}

% title slide
{
    \setbeamertemplate{footline}{} % no page number here
    \frame{
        \titlepage
    }
}






\begin{frame}{Today we're going to cover}
    \vspace{40pt}
    \bi
        \item Greedy Algorithms
        \item Matroids (Recap)
    \ei
\end{frame}


%%%%%%%%%%%%%%%%%%%%%%%%%%%%%%%%%% SLIDE
\begin{frame}{Coin changing}
    \vspace{30pt}
    \bi
        \item We are given a set of coin types: $1$ct, $5$ct, $10$ct, $25$ct, $100$ct
        \item Determine the minimum number $N$ of coins needed to pay out $X $ct!
        \vspace{30pt}
        \visible<2->{
        \item Example: $X=157$ct
        \item Coins: $1\times 100$ct, $2\times 25$ct, $1\times 5$ct, $2 \times 1$ct \quad $\Rightarrow$ \quad $N = 6$
        \visible<3->{\item How do we determine $N$?}
        }
    \ei
\end{frame}

\begin{frame}{Coin changing}
    \vspace{30pt}
    \bi
    	\item 1. Option: Complete Search (via Backtracking) \visible<2->{{\color{gray} -- way to slow}}
    	\visible<3->{\item 2. Option: Dynamic Programming \visible<4->{{\color{gray} -- ok (will be covered next week)\visible<5->{, but what if $N=10^{20}$ and the coins sizes as well?}}}}
    	\visible<6->{\item 3. Option: Greedy!}
    \ei
\end{frame}

\begin{frame}{Cashier's algorithm}
    \vspace{30pt}
    \bi
    	\item While some money is left, pay it with the largest coin possible and repeat.
    	\item<2-> $X=157$ct $\Rightarrow$ take $100$ct, $N = 1$
    	\item<3-> $X=57$ct $\Rightarrow$ take $25$ct, $N = 2$
    	\item<4-> $X=32$ct $\Rightarrow$ take $25$ct, $N = 3$
    	\item<5-> $X=7$ct $\Rightarrow$ take $5$ct, $N = 4$
    	\item<6-> $X=2$ct $\Rightarrow$ take $1$ct, $N = 5$
    	\item<7-> $X=1$ct $\Rightarrow$ take $1$ct, $N = 6$
    	\item<8-> $X=0$ct $\Rightarrow$ done, $N = 6$
    \ei
\end{frame}

\begin{frame}[fragile]{Cashier's algorithm}
\begin{minted}[fontsize=\scriptsize]{cpp}
int coins[N];

int minimumCoins(int X){
    if (!X) return X;
    int i = N-1;
    while (coins[i] > X && i != -1) i--;
    if (i == -1) assert(false);
    return 1 + minimumCoins(X - coins[i]);
}
    \end{minted}
    \vspace{10pt}
    \visible<2->{This can be slow if $X$ is huge and coins are small!}
    \vspace{10pt}
   % \only<2->{
   \pause \pause
\begin{minted}[fontsize=\scriptsize]{cpp}
int minimumCoins(int X){
    if (!X) return X;
    int i = N-1;
    while (coins[i] > X && i != -1) i--;
    if (i == -1) assert(false);
    return X / coins[i] + minimumCoins(X % coins[i]);
}
    \end{minted}
    %}
    \vspace{10pt}
    \visible<4->{Even better versions are possible (e.g. binary search over the coins).}
\end{frame}


\begin{frame}[fragile]{Cashier's algorithm}
    \vspace{30pt}
    \bi
        \item Is the algorithm correct?
        \item<2-> Sometimes ...
        \item<3-> Consider the coins $1$ct, $10$ct, and $25$ct and $X = 40$ct.
        \item<3-> The algorithm will compute $N=7$ ($1\times25$ct, $1\times10$ct, $5\times1$ct)
        \item<3-> The correct answer is $N=4$ ($4\times10$ct)
        \item<4-> What's the difference?
    \ei
\end{frame}

\begin{frame}[fragile]{Cashier's algorithm}
    \vspace{10pt}
    {\textbf{Theorem} The Cashier's algorithm is correct for $1$ct, $5$ct, $10$ct, $25$ct, $100$ct.}\\[\baselineskip]
    {\color{hilight}Proof:}\\[\baselineskip]
    \textit{Observations:} Each solution contains:
    \begin{itemize}
         \item at most four $1$ct coins (else replace with a $5$ct)
         \item at most one $5$ct coins (else replace with a $10$ct)
         \item either two $10$ct or one $5$ct and one $10$ct (else replace with one $25$ct or one $25$ct and one $5$ct)
         \item at most three $25$ct coins (else replace with a $100$ct)
    \end{itemize}
\end{frame}

\begin{frame}[fragile]{Cashier's algorithm}
    \vspace{10pt}
    {\color{hilight}Proof cont'}\\[\baselineskip]
    \begin{tabular}{c|c}
    max. coin & max possible value \\ \hline
    1 & 4 \\
    5 & 9 \\
    10 & 24 \\
    25 & 99 \\
    100 & $\infty$
    \end{tabular}
    \flushleft \qed
      \vspace{30pt}
    \bi
        \item<2-> The algorithm is correct for several sets of coins
        	\bi
        		\item<2-> $1$, $2$, $5$, $10$, $20$, $50$, \dots
        		\item<2-> $1$, $5$, $10$, $50$, \dots
        		\item<2-> Any subset of $\{n^k \mid k \in \mathbb N\}$ for fixed $n$
        	\ei
    \ei
   
\end{frame}


\begin{frame}[fragile]{Matroids}
    \vspace{0pt}
    \bi
        \item Interpret the problem as a more general search problem.
        \item We are given a set of objects $M$ 
%        \bi \item<2-> for CC a multiset of coin instances \ei
        \item<2-> A potential solution is a subset $S \subseteq M$. No two solutions must be subsets of each other.
 %       \bi \item<2-> for CC a solution is a set of coins equal to $X$ct \ei
        \item<3-> Let $\mathcal U = \{A \mid A \subseteq B, \text{ and } B \text{ is a solution}\}$
  %      \bi \item<2-> for CC all possibilities to payout any amount $\leq X$ct \ei
        \item<4-> $(M,\mathcal U)$ is a Matroid iff
        \[ \forall A,B \in \mathcal U: \quad |A| > |B| \Rightarrow \exists x \in A \setminus B: \quad B \cup \{x\} \in \mathcal U \]
   %     \bi \item<2-> for CC al possibilities to payout any amount $\leq X$ct \ei
        \item<5-> We are given a function $f : M \rightarrow \mathbb R ^+$ and are asked to find a solution with minimal $f(S)$
        \item[$\Rightarrow$]<6-> Process $M$ in ascending order of $f$ and maintain a set $S$.
        If an element $m$ can be added to $S$, i.e., if $S \cup \{m\} \in \mathcal U$, then do so.
        \item<7-> This generates the minimal solution.
        \item<8-> Applying it to CC is a {\color{vhilight}bit} complicated ...
    \ei
\end{frame}

\begin{frame}[fragile]{Matroids}
	\vspace{30pt}
	\bi
		\item Proving Matroid property can be very hard
		\item<2-> Instead show the ``greedy property''
		\bi
			\item<2-> For every greedy decision
			\bi
				\item<3-> Assume that you have an optimal solution for the current sub-problem
				\item<4-> Show that it can be transformed into a solution including your greedy choice either with the same (constructive proof) or higher (contradiction) value.
			\ei
		\ei
	\ei
\end{frame}


\begin{frame}[fragile]{Interval Scheduling}
    \vspace{30pt}
    \bi
        \item We are given a set $I$ of intervals $[a_i,b_i]$ 
        \bi \item e.g. a set of jobs to be executed at specific times \ei
        \item<2-> Task: Find the maximum subset $S$ of $I$, s.t., no two intervals in $S$ intersect?
        \bi \item<2-> e.g. find the maximum number of jobs that can be executed \ei
	    \vspace{30pt}
	    \item<3-> Sort intervals by $b_i$.
	    \item<4-> Add intervals to the set of executable intervals
    \ei
\end{frame}

\begin{frame}[fragile]{Interval Scheduling}
    \vspace{30pt}
    \begin{minted}[fontsize=\scriptsize]{cpp}
int maxIntervall(vector<pii> intervals){
    vector<pii> v;
    FORIT(i,intervals) v.push_back(make_pair(i->second, i->first));
    sort(all(v));
    int cl = -oo;
    int c = 0;
    FORIT(i,v)
        if (i->second >= cl) c++, cl = i->first;
    return c;
}
    \end{minted}
\end{frame}

\begin{frame}[fragile]{Interval Scheduling}
    \vspace{30pt}
    {\color{hilight}Correctness:}
    \bi
        \item Let $S$ be a maximal set of intervals.
        \item Since no two intervals intersect, there is a ``fist'' task in $S$.
        \item This task cannot end before the one the greedy algorithm would select.
        \item Thus we can replace the first task in $S$ with this one.
    \ei
    \flushleft\qed
\end{frame}


\begin{frame}[fragile]{Interval Partitioning}
    \vspace{30pt}
    \bi
        \item We are given a set $I$ of intervals $[a_i,b_i]$
        \item<2-> Task: Find the minimal $k$, s.t. I can be partitioned into $k$ set of non-intersecting intervals.
	    \vspace{30pt}
	    \item<3-> Sort intervals by $a_i$.
	    \item<4-> Simulate, add intervals to the partition with the smallest end.
	    If not possible increase $k$
    \ei
\end{frame}


\begin{frame}[fragile]{Interval Partitioning}
    \vspace{30pt}
    \begin{minted}[fontsize=\scriptsize]{cpp}
int minPartition(vector<pii> intervals){
    if (!intervals.size()) return 0;
    sort(all(intervals));

    priority_queue<int> q; q.push(-oo);
    FORIT(i,intervals){
        if (-q.top() <= i->first) q.pop();
        q.push(-i->second);
    }
    return q.size();
}
    \end{minted}
\end{frame}

\begin{frame}[fragile]{Interval Partitioning}
    \vspace{30pt}
    {\color{hilight}Correctness:}
    \bi
    	\item Add interval to already existing partition
    	\bi
    		\item If there is a solution where we don't place the interval in the earliest ending partition, we can swap!
    	\ei
        \item Add a new partition
        \bi
	        \item Since all of the partitions end after the new interval starts, we have no other choice -- all these intervals start before the current one.
        \ei
    \ei
    \flushleft\qed
\end{frame}


\begin{frame}[fragile]{Gale-Shapley}
    \vspace{30pt}
    \bi
        \item Stable Marriage Problem:
        \bi
        	\item Given two sets of objects $A$ and $B$ of the same size.
        	\item For every object we have a preference list of the objects in the other set.
        	\item Determine a pairing, so that to two unpaired object $a$ and $b$ exists, that prefer each other over their current partners.
        \ei
       	\vspace{10pt}
       	\item<2-> Essentially simulate.
       	\item<2-> While there is an unmatched object in $A$, pair it with its preference in $B$ that will accept (being either unpaired or willing to ditch its partner).
       	\item<2-> This algorithm will always terminate (in less than $\mathcal O(n^2)$) with a stable marriage
       	\vspace{10pt}
       	\item<3-> This was worth a Prize named after Alfred Nobel
    \ei
\end{frame}






\end{document}
