\documentclass[12pt,t]{beamer}
\subtitle{06 -- Dynamic Programming I}
\newcommand{\shorttitle}{Dynamic Programming I}
\usepackage{booktabs}
\usepackage{mathtools}
%\usepackage[table]{xcolor}
\usepackage{graphicx}
\setbeameroption{hide notes}
\setbeamertemplate{note page}[plain]

% get rid of junk
\usetheme{default}
\beamertemplatenavigationsymbolsempty
\hypersetup{pdfpagemode=UseNone} % don't show bookmarks on initial view

% font
\usepackage{fontspec}
\setsansfont{TeX Gyre Heros}
\setbeamerfont{note page}{family*=pplx,size=\footnotesize} % Palatino for notes
% "TeX Gyre Heros can be used as a replacement for Helvetica"
% In Unix, unzip the following into ~/.fonts
% In Mac, unzip it, double-click the .otf files, and install using "FontBook"
%   http://www.gust.org.pl/projects/e-foundry/tex-gyre/heros/qhv2.004otf.zip

% named colors
\definecolor{offwhite}{RGB}{249,242,215}
% \definecolor{foreground}{RGB}{255,255,255}
\definecolor{foreground}{RGB}{0,0,0}
% \definecolor{background}{RGB}{24,24,24}
\definecolor{background}{RGB}{255,255,255}
\definecolor{title}{RGB}{107,174,214}
\definecolor{gray}{RGB}{100,100,100}
\definecolor{subtitle}{RGB}{102,155,204}
\definecolor{hilight}{RGB}{20,180,204}
\definecolor{vhilight}{RGB}{255,111,207}
\definecolor{lolight}{RGB}{155,155,155}
%\definecolor{green}{RGB}{125,250,125}

% use those colors
\setbeamercolor{titlelike}{fg=title}
\setbeamercolor{subtitle}{fg=subtitle}
\setbeamercolor{institute}{fg=gray}
\setbeamercolor{normal text}{fg=foreground,bg=background}
\setbeamercolor{item}{fg=foreground} % color of bullets
\setbeamercolor{subitem}{fg=gray}
\setbeamercolor{itemize/enumerate subbody}{fg=gray}
\setbeamertemplate{itemize subitem}{{\textendash}}
\setbeamerfont{itemize/enumerate subbody}{size=\footnotesize}
\setbeamerfont{itemize/enumerate subitem}{size=\footnotesize}

% settings for table of contents
\setbeamercolor{section in toc}{fg=foreground,bg=background}
\setbeamerfont{subsection in toc}{size=\footnotesize}
\setbeamertemplate{section in toc}{{\scriptsize\leavevmode\raise1.35pt\hbox{$\blacktriangleright$}} \inserttocsection}
%\setbeamertemplate{subsection in toc}{\quad{\tiny\leavevmode\raise1.5pt\hbox{$\blacktriangleright$}} \footnotesize\inserttocsubsection\\}
\setbeamertemplate{subsection in toc}{\quad\quad\textendash\enspace\inserttocsubsection\\}

% page number
\setbeamertemplate{footline}{%
    \raisebox{5pt}{\makebox[\paperwidth]{\hfill\makebox[20pt]{\color{gray}
          \scriptsize\insertframenumber}}}\hspace*{5pt}}

% add a bit of space at the top of the notes page
\addtobeamertemplate{note page}{\setlength{\parskip}{12pt}}

% add subsection as frame title when it is empty
\makeatletter
  \CheckCommand*\beamer@checkframetitle{%
    \@ifnextchar\bgroup\beamer@inlineframetitle{}}
  \renewcommand*\beamer@checkframetitle{%
    \global\let\beamer@frametitle\relax\@ifnextchar%
    \bgroup\beamer@inlineframetitle{}}
\makeatother

\addtobeamertemplate{frametitle}{
  \ifx\insertframetitle\empty
      \frametitle{\insertsubsectionhead}
  \else
  \fi
 }{}


\usepackage{pbox}

% a few macros
\newcommand{\bi}{\begin{itemize}}
\newcommand{\ei}{\end{itemize}}
\newcommand{\ig}{\includegraphics}
\newcommand{\subt}[1]{{\footnotesize \color{subtitle} {#1}}}

% title info
\title{Kompetitives Programmieren}
%\author{Gregor Behnke}
\institute{Prof. Dr. Susanne Biundo-Stephan \\ Gregor Behnke \\ Institute of Artificial Intelligence\\ Ulm University}
\date{\tiny based on Bjarki Ágúst Guðmundsson's and Tómas Ken Magnússon's\\Competitive Programming}


\setbeamertemplate{footline}[text line]{%
  \parbox{\linewidth}{\vspace*{-8pt}\shorttitle \hfill \insertframenumber/\inserttotalframenumber}}
\setbeamertemplate{navigation symbols}{}


\newcommand{\specialcell}[2][c]{%
  \begin{tabular}[#1]{@{}c@{}}#2\end{tabular}}

% Tikz
\usepackage{tikz}
\usepackage{tkz-euclide}
\usetikzlibrary{arrows,shapes}
% \usepackage{intersections}
\usetkzobj{all}
\usetikzlibrary{arrows,shapes,angles,quotes,shapes, calc, decorations,matrix}
\usepackage{forest}
\pgfdeclarelayer{bg}    % declare background layer
\pgfsetlayers{bg,main}


% Minted
\usepackage{minted}
\usemintedstyle{tango}
\newminted{cpp}{fontsize=\footnotesize}

% Graph styles
\tikzstyle{vertex}=[circle,fill=black!50,minimum size=15pt,inner sep=0pt, font=\small]
\tikzstyle{selected vertex} = [vertex, fill=red!24]
\tikzstyle{selected2 vertex} = [vertex, fill=hilight!50, text=black]
\tikzstyle{vertex1} = [vertex, fill=red]
\tikzstyle{vertex2} = [vertex, fill=blue]
\tikzstyle{vertex3} = [vertex, fill=green, text=black]
\tikzstyle{vertex4} = [vertex, fill=yellow, text=black]
\tikzstyle{vertex5} = [vertex, fill=pink, text=black]
\tikzstyle{vertex6} = [vertex, fill=purple]
\tikzstyle{edge} = [draw,thick,-]
\tikzstyle{dedge} = [draw,thick,->]
\tikzstyle{weight} = [font=\scriptsize,pos=0.5]
\tikzstyle{selected edge} = [draw,line width=2pt,-,red!50]
\tikzstyle{ignored edge} = [draw,line width=5pt,-,black!20]


\begin{document}

% title slide
{
    \setbeamertemplate{footline}{} % no page number here
    \frame{
        \titlepage
    }
}




\begin{frame}{Today we're going to cover}
    \vspace{40pt}
    \bi
        \item Dynamic Programming \pause -- part I
    \ei
\end{frame}

% TODO: What is dynamic programming?
\begin{frame}{What is dynamic programming?}
    \bi
        \item A problem solving paradigm
        \item Similar in some respects to both divide and conquer and backtracking
        \vspace{5pt}
        \item Divide and conquer recap:
            \bi
                \item Split the problem into \textit{independent} subproblems
                \item Solve each subproblem recursively
                \item Combine the solutions to subproblems into a solution for the given problem
            \ei
        \vspace{5pt}
        \item Dynamic programming:
            \bi
                \item Split the problem into \textit{overlapping} subproblems
                \item Solve each subproblem recursively
                \item Combine the solutions to subproblems into a solution for the given problem
                \item \textit{Don't compute the answer to the same problem more than once}
            \ei
    \ei
\end{frame}

\begin{frame}[fragile]{Top-Down dynamic programming}
    \vspace{30pt}
    \be
        \item Formulate the problem in terms of smaller versions of the problem
        \item Turn this formulation into a recursive function
        \item Memoize the function (remember results that have been computed)
    \ee
\end{frame}

\begin{frame}[fragile]{Top-Down dynamic programming}
    \begin{minted}[fontsize=\footnotesize]{cpp}
map<problem, value> memory;

value dp(problem P) {
    if (is_base_case(P)) {
        return base_case_value(P);
    }

    if (memory.find(P) != memory.end()) {
        return memory[P];
    }

    value result = some value;
    for (problem Q in subproblems(P)) {
        result = combine(result, dp(Q));
    }

    memory[P] = result;
    return result;
}
    \end{minted}
\end{frame}

\begin{frame}[fragile]{Bottom-Up dynamic programming}
    \vspace{30pt}
    \be
        \item Formulate the problem in terms of smaller versions of the problem
        \item Turn this formulation into an \textit{non-recursive} function
        \item \textit{Compute all values by applying the formula iteratively}
    \ee
\end{frame}

\begin{frame}[fragile]{Bottom-Up dynamic programming}
    \begin{minted}[fontsize=\footnotesize]{cpp}
map<state, value> memory;

...
for (state S : base_case)
    memory[S] = base_case_value(S);

for (state S : ordering_of_ordinary_cases){
    value result = some value;
    for (problem Q in subproblems(P)) {
        result = combine(result, memory(Q));
    }
    memory[S] = result;
}
...
    \end{minted}
\end{frame}

\begin{frame}[fragile]{Top-Down vs Bottom-Up DP}
    \vspace{30pt}
    \bi
        \item Top-Down
        \bi
            \item is easier to understand (for beginners)
            \item run-time efficient if not all states are necessary
        \ei
        \item Bottom-Up
        \bi
            \item more difficult to think about (for beginners)
            \item run-time efficient if all states are necessary (no recursive function calls)
            \item often shorter code
        \ei
    \ei
\end{frame}





% TODO: Fibonacci example
\begin{frame}{The Fibonacci sequence}
    \vspace{5pt}
    \textit{The first two numbers in the Fibonacci sequence are 1 and 1. All
            other numbers in the sequence are defined as the sum of the previous two
            numbers in the sequence.}

    \vspace{5pt}
    \bi
        \item Task: Find the $n$th number in the Fibonacci sequence
        \item Let's solve this with dynamic programming
    \ei

    \vspace{5pt}
    \be
        \item Formulate the problem in terms of smaller versions of the problem (recursively)
    \ee

    \begin{align*}
        \mathrm{fibonacci}(1) &= 1\\
        \mathrm{fibonacci}(2) &= 1\\
        \mathrm{fibonacci}(n) &= \mathrm{fibonacci}(n - 2) + \mathrm{fibonacci}(n - 1)
    \end{align*}
\end{frame}

\begin{frame}[fragile]{The Fibonacci sequence}
    \be
        \item[2.] Turn this formulation into a recursive function
    \ee

    \begin{minted}[fontsize=\footnotesize]{cpp}
int fibonacci(int n) {
    if (n <= 2) {
        return 1;
    }

    int res = fibonacci(n - 2) + fibonacci(n - 1);

    return res;
}
    \end{minted}
\end{frame}

\begin{frame}[fragile]{The Fibonacci sequence}
    \bi
        \item What is the time complexity of this? \onslide<2->{Exponential, almost $O(2^n)$}
    \ei

    \begin{figure}

        \begin{tikzpicture}

[-,thick,%
  every node/.style={shape=circle,draw,thick},%
  level distance=0.5cm,
  growth parent anchor={south}, nodes={anchor=north},
  scale=0.8,%
]
\scriptsize
\node {$fib(6)$}
  [sibling distance=5cm]
  child {node {$fib(4)$}
    [sibling distance=2cm]
    child {node {$fib(2)$}
      % [sibling distance=0.5cm]
      % % child {node {$0$}}
      % child {node {$fib(1)$}
      %   % child {node {$0$}}
      %   % child {node {$0$}}
      % }
    }
    child {node {$fib(3)$}
      [sibling distance=1cm]
      child {node {$fib(1)$}
        [sibling distance=0.5cm]
        % child {node {$0$}}
        % child {node {$0$}}
      }
      child {node {$fib(2)$}
        % [sibling distance=0.5cm]
        % % child {node {$0$}}
        % child {node {$fib(1)$}
        %   % child {node {$0$}}
        %   % child {node {$0$}}
        % }
      }
    }
  }
  child {node {$fib(5)$}
    [sibling distance=3cm]
    child {node {$fib(3)$}
      [sibling distance=1cm]
      child {node {$fib(1)$}
        [sibling distance=0.5cm]
        % child {node {$0$}}
        % child {node {$0$}}
      }
      child {node {$fib(2)$}
        % [sibling distance=0.5cm]
        % % child {node {$0$}}
        % child {node {$fib(1)$}
        %   % child {node {$0$}}
        %   % child {node {$0$}}
        % }
      }
    }
    child {node {$fib(4)$}
      [sibling distance=2cm]
      child {node {$fib(2)$}
        % [sibling distance=0.5cm]
        % % child {node {$0$}}
        % child {node {$fib(1)$}
        %   % child {node {$0$}}
        %   % child {node {$0$}}
        % }
      }
      child {node {$fib(3)$}
        [sibling distance=1cm]
        child {node {$fib(1)$}
          [sibling distance=0.5cm]
          % child {node {$0$}}
          % child {node {$0$}}
        }
        child {node {$fib(2)$}
          % [sibling distance=0.5cm]
          % % child {node {$0$}}
          % child {node {$fib(1)$}
          %   % child {node {$0$}}
          %   % child {node {$0$}}
          % }
        }
      }
    }
  };
        \end{tikzpicture}

% \tikzset{
%   treenode/.style = {align=center, inner sep=0pt, text centered,
%     font=\sffamily},
%   vertex/.style = {treenode, circle, white, font=\sffamily\bfseries\tiny, draw=white,
%     text width=1.8em},% arbre rouge noir, noeud noir
% }
% 
% \begin{tikzpicture}
%     [->,>=stealth',level/.style={sibling distance = 5cm/#1,
%   level distance = 1.8cm},scale=0.8] 
%   \node [vertex] {fib(10)}
%     child{ node [vertex] {fib(8)}
%             child{ node [vertex] {fib(6)} 
%                 child{ node [vertex] {fib(4)} }
%                 child{ node [vertex] {fib(5)}}
%             }
%             child{ node [vertex] {fib(7)}
%                             child{ node [vertex] {fib(5)}}
%                             child{ node [vertex] {fib(6)}}
%             }
%     }
%     child{ node [vertex] {fib(9)}
%         child{ node [vertex] {fib(7)} 
%             child{
%                 node [vertex] {fib(5)}
%                     child { node [vertex] {fib(3)} }
%                     child { node [vertex] {fib(4)} }
%                 }
%             child{ node [vertex] {fib(6)}
%             }
%             }
%             child{ node [vertex] {fib(8)}
%                             child{ node [vertex] {49}}
%                             % child{ node [vertex] {}}
%             }
%         }
% ; 
% \end{tikzpicture}
            \end{figure}

\end{frame}

\begin{frame}[fragile]{The Fibonacci sequence}
    \be
        \item[3.] Memoize the function (remember results that have been computed)
    \ee

    \vspace{5pt}

    \begin{minted}[fontsize=\footnotesize]{cpp}
map<int, int> mem;

int fibonacci(int n) {
    if (n <= 2) {
        return 1;
    }

    if (mem.find(n) != mem.end()) {
        return mem[n];
    }

    int res = fibonacci(n - 2) + fibonacci(n - 1);

    mem[n] = res;
    return res;
}
    \end{minted}

\end{frame}

\begin{frame}[fragile]{The Top-Down Fibonacci sequence}
    \vspace{5pt}

    \begin{minted}[fontsize=\footnotesize]{cpp}
int mem[1000];
for (int i = 0; i < 1000; i++)
    mem[i] = -1;

int fibonacci(int n) {
    if (n <= 2) {
        return 1;
    }

    if (mem[n] != -1) {
        return mem[n];
    }

    int res = fibonacci(n - 2) + fibonacci(n - 1);

    mem[n] = res;
    return res;
}
    \end{minted}

\end{frame}

\begin{frame}{The Fibonacci sequence}
    \bi
        \item What is the time complexity now?
        \vspace{5pt}
        \item We have $n$ possible inputs to the function: $1$, $2$, \ldots, $n$.
        \item Each input will either:
            \bi
                \item be computed, and the result saved
                \item be returned from the memory
            \ei
        \item Each input will be computed at most once
        \item Time complexity is $O(n \times f)$, where $f$ is the time complexity of computing an input if we assume that the recursive calls are returned directly from memory ($O(1)$)
        \item Since we're only doing constant amount of work to compute the answer to an input, $f = O(1)$
        \item Total time complexity is $O(n)$
    \ei
\end{frame}

\begin{frame}[fragile]{The Bottom-Up Fibonacci sequence}
    \vspace{5pt}

    \begin{minted}[fontsize=\footnotesize]{cpp}
int mem[1000];

mem[1] = 1;
mem[2] = 1;

for (int i = 3; i < 1000; i++)
    mem[i] = mem[i-1] + mem[i-2];
    
    \end{minted}
\end{frame}



% TODO: Coin change
\begin{frame}{Coin change}
    \vspace{20pt}

    \bi
\item Given an array of coin denominations $d_0$, $d_2$, \ldots, $d_{n-1}$,
            and some amount $x$: What is minimum number of coins needed to
            represent the value $x$?

        %\item Remember the greedy algorithm for Coin change?
        %\item It didn't always give the optimal solution, and sometimes it didn't even give a solution at all...

        %\vspace{10pt}
        %\item What about dynamic programming?
    \ei
\end{frame}

\begin{frame}{Coin change}
    \bi
        \item First step: formulate the problem recursively
        \vspace{20pt}
\item Let $\mathrm{opt}(i,x)$ denote the minimum number of coins needed to represent the value $x$ if we're only allowed to use the coin denominations $d_0$, \ldots, $d_i$
        \vspace{10pt}
        \item Base case: $\mathrm{opt}(i,x) = \infty$ if $x < 0$
        \item Base case: $\mathrm{opt}(i,0) = 0$
        \item Base case: $\mathrm{opt}(-1,x) = \infty$
        \vspace{10pt}
\item $\mathrm{opt}(i,x) = \mathrm{min} \left\{
    \begin{array}{l}
        1 + \mathrm{opt}(i, x - d_i) \\
        \mathrm{opt}(i-1, x)
    \end{array}
\right.$
    \ei
\end{frame}

\begin{frame}[fragile]{Coin change}
    \begin{minted}[fontsize=\footnotesize]{cpp}
int d[10];

int opt(int i, int x) {
    if (x < 0) return oo;
    if (x == 0) return 0;
    if (i == -1) return oo;

    int res = oo;
    res = min(res, 1 + opt(i, x - d[i]));
    res = min(res, opt(i - 1, x));

    return res;
}
    \end{minted}
\end{frame}

\begin{frame}[fragile]{Coin change}
    \begin{minted}[fontsize=\footnotesize]{cpp}
int d[10];
int mem[10][10000];
memset(mem, -1, sizeof(mem));

int opt(int i, int x) {
    if (x < 0) return oo;
    if (x == 0) return 0;
    if (i == -1) return oo;

    if (mem[i][x] != -1) return mem[i][x];

    int res = oo;
    res = min(res, 1 + opt(i, x - d[i]));
    res = min(res, opt(i - 1, x));

    mem[i][x] = res;
    return res;
}
    \end{minted}
\end{frame}

\begin{frame}{Coin change}
    \vspace{30pt}
    \bi
        \item Time complexity?
        \item Number of possible inputs are $n \times x$
        \item Each input will be processed in $O(1)$ time, assuming recursive calls are constant
        \item Total time complexity is $O(n^2)$
    \ei
\end{frame}

\begin{frame}{Coin change}
    \bi
        \vspace{30pt}
        \item How do we know which coins the optimal solution used?
        \item We can store backpointers, or some extra information, to trace backwards through the states
        %\item See example...
    \ei
\end{frame}


\begin{frame}[fragile]{Coin change}
    \bi
        \vspace{30pt}
        \item Is this the end?
        \item<2-> Do we (really) need a 2D array?
        \item<3-> Lets write the code for the bottom-up DP
    \ei
\end{frame}
\begin{frame}[fragile]{Coin change}
    \begin{minted}[fontsize=\footnotesize]{cpp}
int d[10];
int mem[10000];
memset(mem, -1, sizeof(mem));

int opt(int x) {
    if (x < 0) return oo;
    if (x == 0) return 0;

    if (mem[x] != -1) return mem[x];

    int res = oo;
    FOR(i,0,10)
        res = min(res, 1 + opt(x - d[i]));

    mem[x] = res;
    return res;
}
    \end{minted}
\end{frame}

\begin{frame}[fragile]{Coin change}
    \begin{minted}[fontsize=\footnotesize]{cpp}
int d[10]; int mem[10][10000];
FOR(i,0,10) mem[x][0] = 0;

FOR(x,1,10000) FOR(i,0,10) {
    if (j == 0)
        mem[i][x] = oo;
    else 
        mem[i][x] = mem[i-1][x];
    
    if (i - d[i] >= 0)
        mem[i][x] = min(mem[i][x], 1 + mem[i][x-d[x]]);
}
    \end{minted}
    \bi
        \item<2-> Only \texttt{mem[9][i]} is actually relevant!
        \item<3-> \texttt{mem[9][i]} $ = \min_{j=0}^9$ \texttt{mem[j][i]}
        \item<4-> If we remove the first index, do we retain all minimum paths?
    \ei
\end{frame}
\begin{frame}[fragile]{Coin change}
    \begin{minted}[fontsize=\footnotesize]{cpp}
int d[10]; int mem[10000];
mem[0] = 0;

FOR(i,1,10000) {
    mem[i] = oo;    
    FOR(j,0,10) if (i - d[j] >= 0)
        mem[i] = min(mem[i], 1 + mem[i-d[j]]);
}
\end{minted}
\end{frame}


% TODO: Maximum sum
\begin{frame}{Maximum sum}

    \vspace{10pt}

    \bi
\item Given an array $\mathrm{arr}[0]$, $\mathrm{arr}[1]$, \ldots, $\mathrm{arr}[n-1]$ of integers, find the interval with the highest sum
    \ei

    \begin{center}
        \begin{tabular}{|c|c|c|c|c|c|c|}
            \hline
            -15 & \color<2->{vhilight}{8} & \color<2->{vhilight}{-2} & \color<2->{vhilight}{1} & \color<2->{vhilight}{0} & \color<2->{vhilight}{6} & -3 \\
            \hline
        \end{tabular}
    \end{center}

    \bi
        \item<2-> The maximum sum of an interval in this array is $13$

        \item<3-> But how do we solve this in general?
            \bi
        \item Easy to loop through all $\approx n^2$ intervals, and calculate their sums, but that is $O(n^3)$
        \item We could use our static range sum trick to get this down to $O(n^2)$
        \item Can we do better with dynamic programming?
            \ei
    \ei

\end{frame}

\begin{frame}{Maximum sum}

    \vspace{20pt}

    \bi
        \item First step is to formulate this recursively
        \vspace{5pt}
        \item Let $\mathrm{max\_{}sum}(i)$ be the maximum sum interval in the range $0,\ldots,i$
        \vspace{5pt}
        \item Base case: $\mathrm{max\_{}sum}(0) = \mathrm{max}(0, arr[0])$
        \vspace{5pt}
        \item What about $\mathrm{max\_{}sum}(i)$?
        \item What does $\mathrm{max\_{}sum}(i-1)$ return?
        \item Is it possible to combine solutions to subproblems with smaller $i$ into a solution for $i$?
        \vspace{5pt}
        \item At least it's not obvious...
    \ei

\end{frame}

\begin{frame}{Maximum sum}

    \vspace{20pt}

    \bi
        \item Let's try changing perspective
        \vspace{5pt}
    \item Let $\mathrm{max\_{}sum}(i)$ be the maximum sum interval in the range $0,\ldots,i$, \textit{that ends at $i$}
        \vspace{5pt}
        \item Base case: $\mathrm{max\_{}sum}(0) = arr[0]$
        \vspace{5pt}
    \item $\mathrm{max\_{}sum}(i) = \mathrm{max}(arr[i], arr[i] + \mathrm{max\_{}sum}(i - 1))$
        \vspace{15pt}
    \item Then the answer is just $\mathrm{max}\ \{0, \mathrm{max}_{\ 0 \leq i < n}\ \{\ \mathrm{max\_{}sum}(i)\ \}\}$
    \ei

\end{frame}

\begin{frame}[fragile]{Maximum sum}
    \bi
        \item Next step is to turn this into a function
    \ei

    \begin{minted}{cpp}
int arr[1000];

int max_sum(int i) {
    if (i == 0) {
        return arr[i];
    }

    int res = max(arr[i], arr[i] + max_sum(i - 1));

    return res;
}
    \end{minted}
\end{frame}

\begin{frame}[fragile]{Maximum sum}
    \bi
        \item Final step is to memoize the function
    \ei

    \begin{minted}[fontsize=\scriptsize]{cpp}
int arr[1000];
int mem[1000];
bool comp[1000];
memset(comp, 0, sizeof(comp));

int max_sum(int i) {
    if (i == 0) {
        return arr[i];
    }
    if (comp[i]) {
        return mem[i];
    }

    int res = max(arr[i], arr[i] + max_sum(i - 1));

    mem[i] = res;
    comp[i] = true;
    return res;
}
    \end{minted}
\end{frame}

\begin{frame}[fragile]{Maximum sum}
    \bi
        \item Then the answer is just the maximum over all interval ends
    \ei

    \begin{minted}{cpp}
int maximum = 0;
for (int i = 0; i < n; i++) {
    maximum = max(maximum, best_sum(i));
}

printf("%d\n", maximum);
    \end{minted}
\end{frame}

%\begin{frame}[fragile]{Maximum sum}
%    \vspace{40pt}
%    \bi
%        \item If you want to find the maximum sum interval in multiple arrays, remember to clear the memory in between
%    \ei
%\end{frame}

\begin{frame}{Maximum sum}
    \vspace{20pt}
    \bi
        \item What about time complexity?
        \vspace{5pt}
        \item There are $n$ possible inputs to the function
        \item Each input is processed in $O(1)$ time, assuming recursive calls are $O(1)$
        \item Time complexity is $O(n)$
    \ei
\end{frame}



\begin{frame}{2D Max Range Sum}
    \vspace{20pt}
    \bi
      \item Extension of 1D Max Range Sum
      \item Iterate over all lines $z$ as (potential) top-most line of the sub-matrix
      \item Iterate over all lines $x$ as (potential) bottom-most line of the sub-matrix
      \item We can now reduce to 1D Max Range Sum via
    \[a[i] = \sum_{j = z}^x a[i][j]\]
      \item Runtime \pause $\mathcal O(n^4)$ \pause
      \item Clever computation of $a[i]$ gives $\mathcal O(n^3)$
    \ei
\end{frame}

\begin{frame}{2D Max Range Sum}
        \begin{center}
        \begin{tabular}{r|c|c|c|c|c|c|c|}
            \cmidrule{2-7}
            \phantom{$\rightarrow$}&\phantom{mm}&\phantom{mm}&\phantom{mm}&\phantom{mm}&\phantom{mm}&\phantom{mm}\\ \cmidrule{2-7}
            & & & & & & \\ \cmidrule{2-7}
          \visible<2->{$\stackrel{z}{\Rightarrow}$} \visible<3-4>{$\stackrel{x}{\rightarrow}$} & \only<4->{\cellcolor{blue!100}} & \only<4->{\cellcolor{blue!66}} & \only<4->{\cellcolor{blue!33}} & \only<4->{\cellcolor{red!100}} & \only<4->{\cellcolor{red!66}} &\only<4->{\cellcolor{red!33}} \\ \cmidrule{2-7}
           \visible<5-6>{$\stackrel{x}{\rightarrow}$} &\only<6->{\cellcolor{blue!100}} & \only<6->{\cellcolor{blue!66}} & \only<6->{\cellcolor{blue!33}} & \only<6->{\cellcolor{red!100}} & \only<6->{\cellcolor{red!66}} &\only<6->{\cellcolor{red!33}} \\ \cmidrule{2-7}
           \visible<7-8>{$\stackrel{x}{\rightarrow}$} &\only<8->{\cellcolor{blue!100}} & \only<8->{\cellcolor{blue!66}} & \only<8->{\cellcolor{blue!33}} & \only<8->{\cellcolor{red!100}} & \only<8->{\cellcolor{red!66}} &\only<8->{\cellcolor{red!33}} \\ \cmidrule{2-7}
           \visible<9-10>{$\stackrel{x}{\rightarrow}$} &\only<10->{\cellcolor{blue!100}} & \only<10->{\cellcolor{blue!66}} & \only<10->{\cellcolor{blue!33}} & \only<10->{\cellcolor{red!100}} & \only<10->{\cellcolor{red!66}} &\only<10->{\cellcolor{red!33}} \\ \cmidrule{2-7}
           \visible<11-12>{$\stackrel{x}{\rightarrow}$} &\only<12->{\cellcolor{blue!100}} & \only<12->{\cellcolor{blue!66}} & \only<12->{\cellcolor{blue!33}} & \only<12->{\cellcolor{red!100}} & \only<12->{\cellcolor{red!66}} &\only<12->{\cellcolor{red!33}} \\ \cmidrule{2-7}
           \visible<13-14>{$\stackrel{x}{\rightarrow}$} &\only<14->{\cellcolor{blue!100}} & \only<14->{\cellcolor{blue!66}} & \only<14->{\cellcolor{blue!33}} & \only<14->{\cellcolor{red!100}} & \only<14->{\cellcolor{red!66}} &\only<14->{\cellcolor{red!33}} \\ \cmidrule{2-7}
        \end{tabular}
        \end{center}
\end{frame}

\begin{frame}[fragile]{2D Max Range Sum}
    \begin{minted}[fontsize=\footnotesize]{cpp}
int array[101][101];
int maxSum2D(int N){
    int S = 0;
    int pr[100];
    FOR(z, 0, N){
        FOR(i, 0, N) pr[i] = 0;
        FOR(x, z, N){
            int t = 0, s = -oo;
            FOR(i, 0, N){
                pr[i] = pr[i] + array[x][i];
                t = t + pr[i];
                if (t > s) s = t;
                if (t < 0) t = 0;
            }
            if(s > S) S = s;
        }
    }
    return S;
}
    \end{minted}
\end{frame}


\begin{frame}[fragile]{Coin change -- Revisited}
    \bi
        \vspace{10pt}
        \item What if we want to count how many ways there are to pay out a certain amount?
        \item<2-> We should not count $1ct,2ct$ and $2ct,1ct$ as different ways.
        \item<3-> We are counting only \emph{sorted} ways to pay out!
        \vspace{10pt}
\item<4-> Let $\mathrm{opt}(i,x)$ denote the number of possibilities to represent the value $x$ if we're only allowed to use the coin denominations $d_0$, \ldots, $d_i$ \emph{and the last coin was $d_i$}
        \vspace{10pt}
        \item<5-> Base case: $\mathrm{opt}(i,x) = 0$ if $x < 0$
        \item<5-> Base case: $\mathrm{opt}(i,0) = 1$
        \item<5-> Base case: $\mathrm{opt}(-1,x) = 0$
        \vspace{10pt}
        \item<5-> $\mathrm{opt}(i,x) = \sum_{j=0}^i \mathrm{opt}(j, x - d_i)$
    \ei
\end{frame}
\begin{frame}[fragile]{Coin change -- Revisited}
    \begin{minted}[fontsize=\footnotesize]{cpp}
int d[10]; int mem[10][10000];
int res[10000];
FOR(i,1,10) mem[i][0] = 0;
mem[0][0] = res[0] = 1;

FOR(i,1,10000) {
    FOR(j,0,10) {
        mem[j][i] = 0;
        if (d[j] > i) continue;
        FOR(k,0,j+1)
            mem[j][i] += mem[k][i - d[j]];
    }
    res[i] = 0;
    FOR(j,0,10) res[i] += mem[j][i];
}
    \end{minted}
    \bi
        \item<2-> Here we can't get rid of the second dimension
    \ei
\end{frame}


%\begin{frame}{Squares}
%    \vspace{20pt}
%    \bi
%      \item Determine the smallest $n$, s.t. $N$ is the sum of $n$ squares
%      \item Use DP over $(i,j)$  -- number of squares necessary to achieve $i$ where the largest square is $j^2$.
%      \item Runtime \pause $\mathcal O(N\cdot\sqrt{N}^2) = \mathcal O(N^2)$ \pause
%      \item Do we really need $j$?
%      \item DP over $(i)$ number of squares necessary to achieve $i$ and 
%      \[
%    dp(i) = \min_{j = 0}^{\sqrt{i}} 1 + dp(i - j^2)
%      \]
%      \item Runtime \pause $\mathcal O(N\cdot\sqrt{N})$
%    \ei
%\end{frame}
%
%\begin{frame}[fragile]{Squares}
%    \begin{minted}[fontsize=\footnotesize]{cpp}
%int dp[10001];
%
%dp[0] = 0;
%FOR(i,1,10001){
%    int x = 1;
%    dp[i] = oo;
%    while (x*x <= i) dp[i] = min(dp[i], dp[i-x*x]+1), x++;
%}
%    \end{minted}
%\end{frame}
%
% TODO: Longest common subsequence
\begin{frame}{Longest common subsequence}
    \vspace{20pt}
    \bi
\item Given two strings (or arrays of integers) $a[0]$, \ldots, $a[n-1]$ and $b[0]$, \ldots, $b[m-1]$, find the length of the longest subsequence that they have in common.

    \vspace{10pt}
\item $a = $\texttt{"b\underline{an}an\underline{inn}"}
\item $b = $\texttt{"k\underline{anin}a\underline{n}"}
    \vspace{5pt}
\item The longest common subsequence of $a$ and $b$, \texttt{"aninn"}, has length 5
    \ei
\end{frame}

\begin{frame}{Longest common subsequence}
    \vspace{20pt}
    \bi
\item Let $\mathrm{lcs}(i, j)$ be the length of the longest common subsequence of the strings $a[0]$, \ldots, $a[i]$ and $b[0]$, \ldots, $b[j]$

    \vspace{10pt}
\item Base case: $\mathrm{lcs}(-1, j) = 0$
\item Base case: $\mathrm{lcs}(i, -1) = 0$
    \vspace{10pt}
\item $\mathrm{lcs}(i, j) = \mathrm{max} \left\{
    \begin{array}{ll}
        \mathrm{lcs}(i,j-1) & \\
        \mathrm{lcs}(i-1,j) & \\
        1 + \mathrm{lcs}(i-1,j-1) & \textrm{if } a[i] = b[j] \\
    \end{array}
\right.$
    \ei
\end{frame}

\begin{frame}[fragile]{Longest common subsequence}
    \begin{minted}[fontsize=\scriptsize]{cpp}
string a = "bananinn",
       b = "kaninan";
int mem[1000][1000];
memset(mem, -1, sizeof(mem));

int lcs(int i, int j) {
    if (i == -1 || j == -1) {
        return 0;
    }
    if (mem[i][j] != -1) {
        return mem[i][j];
    }

    int res = 0;
    res = max(res, lcs(i, j - 1));
    res = max(res, lcs(i - 1, j));

    if (a[i] == b[j]) {
        res = max(res, 1 + lcs(i - 1, j - 1));
    }

    mem[i][j] = res;
    return res;
}
    \end{minted}
\end{frame}

\begin{frame}{Longest common subsequence}
    \vspace{40pt}
    \bi
        \item Time complexity? \pause
            \vspace{10pt}
        \item There are $n\times m$ possible inputs
        \item Each input is processed in $O(1)$, assuming recursive calls are $O(1)$
        \item Total time complexity is $O(n\times m)$
    \ei
\end{frame}


\begin{frame}{Wedding Shopping}
    \vspace{20pt}
    \bi
        \item Given a set of garments $G$ where each garment has a (integer) price $p_i$ and a type $t_i$.
        \item We have to buy one garment of each type and spend as much as possible, but not more than $M$.
    \ei
    \pause
    An example ($M = \only<2-3>{20}\only<4>{\mathbf{17}}$ and $C = 3$)
    \bi
        \item 1. type: 6 \only<2-3>{4}\only<4>{\textbf{4}} \only<2,4>{8}\only<3>{\textbf{8}}
        \item 2. type: 5 \only<2>{10}\only<3-4>{\textbf{10}}
        \item 3. type: \only<2,4>{1}\only<3>{\textbf{1}} 5 \only<2-3>{3}\only<4>{\textbf{3}} 5
    \ei
\end{frame}

\begin{frame}{Wedding Shopping}
    %\vspace{10pt}
    \bi[<+->]
        \item Let $max\_spent(i)$ be the max amount we can spend to buy the first $i$ garments
        \item $max\_spent(i) = $\\$\max_{g} max\_spent(i-1) + p_g$ and not $\geq M$
        \item \dots essentially greedy \only<4->{...  and not correct}
        \item<5-> add the amount of spend money as a dimension
        \item<5-> the DP-value is now a bool
        \item<5-> $spent(i,v) = \bigvee_g spent(i-1,v-p_g)$
        \item<6->[$\Rightarrow$] similar to 0-1 Knapsack
    \ei
    \visible<4>{
    $M = 10$ and $C = 3$
    \bi
        \item 1. type: 1 2 10
        \item 2. type: 4
        \item 3. type: 4
    \ei
    }
\end{frame}


\begin{frame}[fragile]{Wedding Shopping}
    \begin{minted}[fontsize=\footnotesize]{cpp}
vi garms[20];
bool dp[21][201];

FOR(i,0,m) dp[0][i] = false;
dp[0][m] = true;
    
FOR(gt,1,c+1) FOR(a,0,m+1){
    dp[gt][a] = false;
    FORIT(g,garms[gt-1])
        if (a + *g <= m && dp[gt-1][a+*g]) dp[gt][a] = true;
}
bool found = false;
FOR(i,0,m+1) if (dp[c][i]) {
    cout << m-i << endl;
    found = true;
    break;
}
if (!found) cout << "no solution" << endl;
    \end{minted}
\end{frame}


\end{document}
