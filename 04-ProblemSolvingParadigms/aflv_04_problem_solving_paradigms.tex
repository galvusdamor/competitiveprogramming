\documentclass[12pt,t]{beamer}
\subtitle{04 -- Problem Solving Paradigms}
\newcommand{\shorttitle}{Problem Solving Paradigms}
\usepackage{graphicx}
\setbeameroption{hide notes}
\setbeamertemplate{note page}[plain]

% get rid of junk
\usetheme{default}
\beamertemplatenavigationsymbolsempty
\hypersetup{pdfpagemode=UseNone} % don't show bookmarks on initial view

% font
\usepackage{fontspec}
\setsansfont{TeX Gyre Heros}
\setbeamerfont{note page}{family*=pplx,size=\footnotesize} % Palatino for notes
% "TeX Gyre Heros can be used as a replacement for Helvetica"
% In Unix, unzip the following into ~/.fonts
% In Mac, unzip it, double-click the .otf files, and install using "FontBook"
%   http://www.gust.org.pl/projects/e-foundry/tex-gyre/heros/qhv2.004otf.zip

% named colors
\definecolor{offwhite}{RGB}{249,242,215}
% \definecolor{foreground}{RGB}{255,255,255}
\definecolor{foreground}{RGB}{0,0,0}
% \definecolor{background}{RGB}{24,24,24}
\definecolor{background}{RGB}{255,255,255}
\definecolor{title}{RGB}{107,174,214}
\definecolor{gray}{RGB}{100,100,100}
\definecolor{subtitle}{RGB}{102,155,204}
\definecolor{hilight}{RGB}{20,180,204}
\definecolor{vhilight}{RGB}{255,111,207}
\definecolor{lolight}{RGB}{155,155,155}
%\definecolor{green}{RGB}{125,250,125}

% use those colors
\setbeamercolor{titlelike}{fg=title}
\setbeamercolor{subtitle}{fg=subtitle}
\setbeamercolor{institute}{fg=gray}
\setbeamercolor{normal text}{fg=foreground,bg=background}
\setbeamercolor{item}{fg=foreground} % color of bullets
\setbeamercolor{subitem}{fg=gray}
\setbeamercolor{itemize/enumerate subbody}{fg=gray}
\setbeamertemplate{itemize subitem}{{\textendash}}
\setbeamerfont{itemize/enumerate subbody}{size=\footnotesize}
\setbeamerfont{itemize/enumerate subitem}{size=\footnotesize}

% settings for table of contents
\setbeamercolor{section in toc}{fg=foreground,bg=background}
\setbeamerfont{subsection in toc}{size=\footnotesize}
\setbeamertemplate{section in toc}{{\scriptsize\leavevmode\raise1.35pt\hbox{$\blacktriangleright$}} \inserttocsection}
%\setbeamertemplate{subsection in toc}{\quad{\tiny\leavevmode\raise1.5pt\hbox{$\blacktriangleright$}} \footnotesize\inserttocsubsection\\}
\setbeamertemplate{subsection in toc}{\quad\quad\textendash\enspace\inserttocsubsection\\}

% page number
\setbeamertemplate{footline}{%
    \raisebox{5pt}{\makebox[\paperwidth]{\hfill\makebox[20pt]{\color{gray}
          \scriptsize\insertframenumber}}}\hspace*{5pt}}

% add a bit of space at the top of the notes page
\addtobeamertemplate{note page}{\setlength{\parskip}{12pt}}

% add subsection as frame title when it is empty
\makeatletter
  \CheckCommand*\beamer@checkframetitle{%
    \@ifnextchar\bgroup\beamer@inlineframetitle{}}
  \renewcommand*\beamer@checkframetitle{%
    \global\let\beamer@frametitle\relax\@ifnextchar%
    \bgroup\beamer@inlineframetitle{}}
\makeatother

\addtobeamertemplate{frametitle}{
  \ifx\insertframetitle\empty
      \frametitle{\insertsubsectionhead}
  \else
  \fi
 }{}


\usepackage{pbox}

% a few macros
\newcommand{\bi}{\begin{itemize}}
\newcommand{\ei}{\end{itemize}}
\newcommand{\ig}{\includegraphics}
\newcommand{\subt}[1]{{\footnotesize \color{subtitle} {#1}}}

% title info
\title{Kompetitives Programmieren}
%\author{Gregor Behnke}
\institute{Prof. Dr. Susanne Biundo-Stephan \\ Gregor Behnke \\ Institute of Artificial Intelligence\\ Ulm University}
\date{\tiny based on Bjarki Ágúst Guðmundsson's and Tómas Ken Magnússon's\\Competitive Programming}


\setbeamertemplate{footline}[text line]{%
  \parbox{\linewidth}{\vspace*{-8pt}\shorttitle \hfill \insertframenumber/\inserttotalframenumber}}
\setbeamertemplate{navigation symbols}{}


\newcommand{\specialcell}[2][c]{%
  \begin{tabular}[#1]{@{}c@{}}#2\end{tabular}}

% Tikz
\usepackage{tikz}
\usepackage{tkz-euclide}
\usetikzlibrary{arrows,shapes}
% \usepackage{intersections}
\usetkzobj{all}
\usetikzlibrary{arrows,shapes,angles,quotes,shapes, calc, decorations,matrix}
\usepackage{forest}
\pgfdeclarelayer{bg}    % declare background layer
\pgfsetlayers{bg,main}


% Minted
\usepackage{minted}
\usemintedstyle{tango}
\newminted{cpp}{fontsize=\footnotesize}

% Graph styles
\tikzstyle{vertex}=[circle,fill=black!50,minimum size=15pt,inner sep=0pt, font=\small]
\tikzstyle{selected vertex} = [vertex, fill=red!24]
\tikzstyle{selected2 vertex} = [vertex, fill=hilight!50, text=black]
\tikzstyle{vertex1} = [vertex, fill=red]
\tikzstyle{vertex2} = [vertex, fill=blue]
\tikzstyle{vertex3} = [vertex, fill=green, text=black]
\tikzstyle{vertex4} = [vertex, fill=yellow, text=black]
\tikzstyle{vertex5} = [vertex, fill=pink, text=black]
\tikzstyle{vertex6} = [vertex, fill=purple]
\tikzstyle{edge} = [draw,thick,-]
\tikzstyle{dedge} = [draw,thick,->]
\tikzstyle{weight} = [font=\scriptsize,pos=0.5]
\tikzstyle{selected edge} = [draw,line width=2pt,-,red!50]
\tikzstyle{ignored edge} = [draw,line width=5pt,-,black!20]


\begin{document}

% title slide
{
    \setbeamertemplate{footline}{} % no page number here
    \frame{
        \titlepage
    }
}





\begin{frame}{Today we're going to cover}
    \vspace{40pt}
    \bi
        \item Problem solving paradigms
        \item Complete search
        \item Backtracking
        \item Divide and conquer
    \ei
\end{frame}

%\begin{frame}{Example problem}
%    \bi
%        \item Problem C from NWERC 2006: Pie
%    \ei
%\end{frame}

% TODO: Problem solving paradigms, what is that?
\begin{frame}{Problem solving paradigms}
    \vspace{40pt}
    \bi
        \item What is a problem solving paradigm?
        \item A method to construct a solution to a specific type of problem
        % \item One example is "Brute force"
        %     \bi
        %         \item the problem asks us to find an object with a specific property (\textit{the problem type})
        %         \item apply the "Brute force" paradigm: go through all objects, for each of them check if the object has the property, if yes, stop, if no, continue
        %     \ei
        % \vspace{10pt}
		\item Today and in later (and previous) lectures we will study common problem solving paradigms
        % \item Each paradigm applies to different kinds of problems
    \ei
\end{frame}

% % todo: problem solving paradigms, overview (greedy, dp, ...)
% \begin{frame}{Problem solving paradigms}
%     \bi
%         \item Complete search
%     \ei
% \end{frame}

% todo: complete search
\begin{frame}{Complete search}
    \vspace{5pt}
    \bi
        \item We have a finite set of objects
        \item We want to find an element in that set which satisfies some constraints
            \bi
                \item or find \textbf{all} elements in that set which satisfy some constraints
            \ei

        \vspace{5pt}
        \item Simple! Just go through all elements in the set, and for each of them check if they satisify the constraints
        \item Of course it's not going to be very efficient...
        \item But remember, we always want the simplest solution that runs in time
        \item Complete search should be the first problem solving paradigm you think about when you're trying to solve a problem
    \ei
\end{frame}

% \begin{frame}{Example problem: Vito's family}
%     \bi
%         \item http://uva.onlinejudge.org/external/100/10041.html
%     \ei
% \end{frame}

% todo: tools: next_permutation, loop through subsets with bitmask
\begin{frame}{Complete search}
    \vspace{20pt}
    \bi
        \item What if the search space is more complex?
            \bi
                \item All permutations of $n$ items
                \item All subsets of $n$ items
                \item All ways to put $n$ queens on an $n\times n$ chessboard without any queen attacking any other queen
            \ei
        \item How are we supposed to iterate through the search space?
        \item Let's take a better look at these examples
    \ei
\end{frame}

\begin{frame}[fragile]{Iterating through permutations}
    \bi
        \item Already implemented in many standard libraries:
            \bi
                \item \texttt{next\_{}permutation} in C++
                \item \texttt{itertools.permutations} in Python
            \ei
    \ei

            \begin{minted}{cpp}
int n = 5;
vector<int> perm(n);
for (int i = 0; i < n; i++) perm[i] = i + 1;

do {
    for (int i = 0; i < n; i++) {
        printf("%d ", perm[i]);
    }
    printf("\n");

} while (next_permutation(perm.begin(), perm.end()));
            \end{minted}

\end{frame}

\begin{frame}{Iterating through permutations}
    \vspace{40pt}

		\bi
        %\item Even simpler in Python...
       \vspace{20pt}
	   \item Remember that there are $n!$ permutations of length $n$, so usually you can only go through all permutations if $n \leq 11$
            \bi
                \item Otherwise you need to find a more clever approach than complete search
            \ei
           \vspace{20pt}
    \ei
\end{frame}

% TODO: Example problem

\begin{frame}[fragile]{Iterating through subsets}
    \bi
        \item Remember the bit representation of sets?
        \item Each integer from $0$ to $2^n - 1$ represents a different subset of the set $\{1,2,\ldots,n\}$
        \item Just iterate through the integers
    \ei

            \begin{minted}{cpp}
int n = 5;
for (int subset = 0; subset < (1 << n); subset++) {
    for (int i = 0; i < n; i++) {
        if ((subset & (1 << i)) != 0) {
            printf("%d ", i+1);
        }
    }
    printf("\n");
}
            \end{minted}
\end{frame}

\begin{frame}{Iterating through subsets}
    \vspace{40pt}

    \bi
       % \item Similar in Python
        \vspace{20pt}
        \item Remember that there are $2^n$ permutations of length $n$, so usually you can only go through all permutations if $n \leq 25$
            \bi
                \item Otherwise you need to find a more clever approach than complete search
            \ei
            \vspace{20pt}
    \ei
\end{frame}

% TODO: Example problem

% TODO: Backtracking
\begin{frame}[fragile]{Backtracking}
    \vspace{30pt}

    \bi
        \item We've seen two ways to go through a complex search space, but both of the solutions were rather specific
        \item Would be nice to have a more general ``framework''
        \vspace{10pt}
        \item Backtracking!
    \ei
\end{frame}

\begin{frame}[fragile]{Backtracking}
    \vspace{20pt}
    \bi
        \item Define states
            \bi
                \item We have one initial ``empty'' state
                \item Some states are partial
                \item Some states are complete
            \ei
        \vspace{10pt}
        \item Define transitions from a state to possible next states
        \vspace{10pt}
        \item Basic idea:
            \begin{enumerate}
                \item Start with the empty state
                \item Use recursion to traverse all states by going through the transitions
                \item If the current state is invalid, then stop exploring this branch
                \item Process all complete states (these are the states we're looking for)
            \end{enumerate}
    \ei
\end{frame}


\begin{frame}[fragile]{Backtracking}
    \vspace{10pt}
    \bi
        \item General solution form:
    \ei

    \begin{minted}[fontsize=\scriptsize]{cpp}
state S;

void generate() {
    if (!is_valid(S))
        return;

    if (is_complete(S))
        print(S);

    foreach (possible next move P) {
        apply move P;
        generate();
        undo move P;
    }
}

S = empty state;
generate();
    \end{minted}
\end{frame}


% TODO: Generating all subsets
\begin{frame}[fragile]{Generating all subsets}
    \bi
        \item Also simple to do with backtracking:
    \ei
    \begin{minted}[fontsize=\tiny]{cpp}
const int n = 5;
bool pick[n];

void generate(int at) {
    if (at == n) {
        for (int i = 0; i < n; i++) {
            if (pick[i]) {
                printf("%d ", i+1);
            }
        }
        printf("\n");
    } else {

        // either pick element no. at
        pick[at] = true;
        generate(at + 1);

        // or don't pick element no. at
        pick[at] = false;
        generate(at + 1);
    }
}

generate(0);
    \end{minted}
\end{frame}


% TODO: Generating all permutations
\begin{frame}[fragile]{Generating all permutations}
    \bi
        \item Also simple to do with backtracking:
    \ei
    \begin{minted}[fontsize=\tiny]{cpp}
const int n = 5;
int perm[n];
bool used[n];

void generate(int at) {
    if (at == n) {
        for (int i = 0; i < n; i++) {
            printf("%d ", perm[i]+1);
        }
        printf("\n");
    } else {

        // decide what the at-th element should be
        for (int i = 0; i < n; i++) {
            if (!used[i]) {
                used[i] = true;
                perm[at] = i;

                generate(at + 1);

                // remember to undo the move:
                used[i] = false;
            }
        }
    }
}

memset(used, 0, n);
generate(0);
    \end{minted}
\end{frame}

\begin{frame}[fragile]{$n$ queens}
    \vspace{20pt}
    \bi
        \item Given $n$ queens and an $n\times n$ chessboard, find all ways to
            put the $n$ queens on the chessboard such that no queen can attack
            any other queen

        \vspace{10pt}

        \item This is a very specific set we want to iterate through, so we probably won't find this in the standard library
        \item We could use our bit trick to iterate through all subsets of the $n\times n$ cells of size $n$, but that would be very slow
        \vspace{20pt}
        \item Let's use backtracking
    \ei
\end{frame}

\begin{frame}[fragile]{$n$ queens}
    \vspace{10pt}
    \bi
        \item Go through the cells in increasing order
        \item Either put a queen on that cell or not (transition)
        \item Don't put down a queen if she's able to attack another queen already on the table
    \ei

    \vspace{10pt}

    \begin{minted}{cpp}
const int n = 8;
bool has_queen[n][n];
int queens_left = n;

// generate function

memset(has_queen, 0, sizeof(has_queen));
generate(0, 0);
    \end{minted}
\end{frame}

\begin{frame}[fragile]{$n$ queens}
    \begin{minted}[fontsize=\tiny]{cpp}
void generate(int x, int y) {
    if (y == n) {
        generate(x+1, 0);
    } else if (x == n) {
        if (queens_left == 0) {
            for (int i = 0; i < n; i++) {
                for (int j = 0; j < n; j++) {
                    printf("%c", has_queen[i][j] ? 'Q' : '.');
                }
                printf("\n");
            }
        }
    } else {

        if (queens_left > 0 and no queen can attack cell (x,y)) {
            // try putting a queen on this cell
            has_queen[x][y] = true;
            queens_left--;

            generate(x, y+1);

            // undo the move
            has_queen[x][y] = false;
            queens_left++;
        }

        // try leaving this cell empty
        generate(x, y+1);
    }
}
    \end{minted}
\end{frame}

% TODO: Example problem
% \begin{frame}{Example problem: The Hamming Distance Problem}
%     \bi
%         \item http://uva.onlinejudge.org/external/7/729.html
%     \ei
% \end{frame}

% TODO: Divide and conquer
\begin{frame}{Divide and conquer}
    \vspace{10pt}
    \bi
        \item Given an instance of the problem, the basic idea is to
            \begin{enumerate}
                \item split the problem into one or more smaller subproblems
                \item solve each of these subproblems recursively
                \item combine the solutions to the subproblems into a solution of the given problem
            \end{enumerate}

        \vspace{10pt}

        \item Some standard divide and conquer algorithms:
            \bi
                \item Quicksort
                \item Mergesort
                \item Karazuba algorithm
                \item Strassen algorithm
                \item Many algorithms from computational geometry
                    \bi
                        \item Convex hull
                        \item Closest pair of points
                    \ei
            \ei
    \ei
\end{frame}

\begin{frame}[fragile]{Divide and conquer: Time complexity}
    \begin{minted}[fontsize=\scriptsize]{cpp}
void solve(int n) {
    if (n == 0)
        return;

    solve(n/2);
    solve(n/2);

    for (int i = 0; i < n; i++) {
        // some constant time operations
    }
}
    \end{minted}

    \bi
        \item What is the time complexity of this divide and conquer algorithm?
        \item Usually helps to model the time complexity as a recurrence relation:
            \bi
                \item $T(n) = 2T(n/2) + n$
            \ei
    \ei
\end{frame}

\begin{frame}[fragile]{Divide and conquer: Time complexity}
    \bi
        \item But how do we solve such recurrences?
        \item Usually simplest to use the Master theorem when applicable
            \bi
                \item It gives a solution to a recurrence of the form $T(n) = aT(n/b) + f(n)$ in asymptotic terms
                \item All of the divide and conquer algorithms mentioned so far have a recurrence of this form
            \ei
        \vspace{10pt}
        \item The Master theorem tells us that $T(n) = 2T(n/2) + n$ has asymptotic time complexity $O(n \log n)$
        \vspace{10pt}
        \item You don't need to know the Master theorem for this course, but still recommended as it's very useful
    \ei
\end{frame}

\begin{frame}{Decrease and conquer}
    \vspace{30pt}
    \bi
        \item Sometimes we're not actually dividing the problem into many subproblems, but only into one smaller subproblem
        \item Usually called decrease and conquer
        \item The most common example of this is binary search
    \ei
\end{frame}

% TODO: Example problem
% TODO: Other types of divide and conquer
\begin{frame}{Divide and conquer}
    \vspace{20pt}
    \bi
        \item Binary search is very useful, can be used to construct simple and efficient solutions to problems
        \item But binary search is only one example of divide and conquer
        \item Let's explore two more examples
    \ei
\end{frame}

% TODO: Binary exponentiation
\begin{frame}[fragile]{Binary exponentiation}
    \bi
        \item We want to calculate $x^n$, where $x,n$ are integers
        \item Assume we don't have the built in \texttt{pow} method
        \item Naive method:
    \ei

    \begin{minted}{cpp}
int pow(int x, int n) {
    int res = 1;
    for (int i = 0; i < n; i++) {
        res = res * x;
    }

    return res;
}
    \end{minted}

    \bi
        \item This is $O(n)$, but what if we want to support large $n$ efficiently?
    \ei
\end{frame}

\begin{frame}{Binary exponentiation}
    \bi
        \item Let's use divide and conquer
        \vspace{10pt}
        \item Notice the three identities:

            \bi
                \item $x^0 = 1$
                \item $x^n = x \times x^{n-1}$
                \item $x^n = x^{n/2} \times x^{n/2}$
            \ei

        \item Or in terms of our function:

            \bi
                \item $pow(x,0) = 1$
                \item $pow(x,n) = x \times pow(x, n-1)$
                \item $pow(x,n) = pow(x, n/2) \times pow(x, n/2)$
            \ei

        \item $pow(x,n/2)$ is used twice, but we only need to compute it once:

            \bi
                \item $pow(x,n) = pow(x, n/2)^2$
            \ei
    \ei
\end{frame}

\begin{frame}[fragile]{Binary exponentiation}
    \vspace{20pt}

    \bi
        \item Let's try using these identities to compute the answer recursively
    \ei

    \vspace{10pt}

    \begin{minted}{cpp}
int pow(int x, int n) {
    if (n == 0) return 1;
    return x * pow(x, n - 1);
}
    \end{minted}

    \vspace{10pt}

    \bi
        \item<2-> How efficient is this?
            \bi
                \item $T(n) = 1 + T(n-1)$
                \item<3-> $O(n)$
                \item<4-> Still just as slow...
            \ei
    \ei

\end{frame}

\begin{frame}[fragile]{Binary exponentiation}
    \bi
        \item What about the third identity?
            \bi
                \item $n/2$ is not an integer when $n$ is odd, so let's only use it when $n$ is even
            \ei
    \ei

    \begin{minted}{cpp}
int pow(int x, int n) {
    if (n == 0) return 1;
    if (n % 2 != 0) return x * pow(x, n - 1);
    int st = pow(x, n/2);
    return st * st;
}
    \end{minted}

    \bi
        \item How efficient is this?
            \bi
                \item<2-> $T(n) = 1 + T(n-1)$ if $n$ is odd
                \item<2-> $T(n) = 1 + T(n/2)$ if $n$ is even
                \item<3-> Since $n-1$ is even when $n$ is odd:
                \item<3-> $T(n) = 1 + 1 + T((n-1)/2)$ if $n$ is odd
                \item<4-> $O(\log n)$
                \item<4-> Fast!
            \ei
    \ei
\end{frame}

\begin{frame}{Binary exponentiation}
    \vspace{20pt}
    \bi
        \item Notice that $x$ doesn't have to be an integer, and $\star$ doesn't have to be integer multiplication...
        \item It also works for:
            \bi
                \item Computing $x^n$, where $x$ is a floating point number and $\star$ is floating point number multiplication
                \item Computing $A^n$, where $A$ is a matrix and $\star$ is matrix multiplication
                \item Computing $x^n \pmod{m}$, where $x$ is a matrix and $\star$ is integer multiplication modulo $m$
                \item Computing $x\star x\star \cdots \star x$, where $x$ is any element and $\star$ is any associative operator
            \ei

        \item All of these can be done in $O(\log(n) \times f)$, where $f$ is the cost of doing one application of the $\star$ operator
    \ei
\end{frame}

\begin{frame}{Fibonacci words}
    \bi
        \item Recall that the Fibonacci sequence can be defined as follows:
            \bi
        \item $\mathrm{fib}_1 = 1$
        \item $\mathrm{fib}_2 = 1$
        \item $\mathrm{fib}_n = \mathrm{fib}_{n-2} + \mathrm{fib}_{n-1}$
            \ei
        \item We get the sequence $1, 1, 2, 3, 5, 8, 13, 21, \ldots$
        \vspace{10pt}
        \item There are many generalizations of the Fibonacci sequence
        \item One of them is to start with other numbers, like:
            \bi
                \item $f_1 = 5$
                \item $f_2 = 4$
                \item $f_n = f_{n-2} + f_{n-1}$
            \ei
        \item We get the sequence $5, 4, 9, 13, 22, 35, 57, \ldots$
        \vspace{10pt}
        \item What if we start with something other than numbers?
    \ei
\end{frame}

\begin{frame}{Fibonacci words}
    \vspace{10pt}
    \bi
        \item Let's try starting with a pair of strings, and let $+$ denote string concatenation:
            \bi
                \item $g_1 = A$
                \item $g_2 = B$
                \item $g_n = g_{n-2} + g_{n-1}$
            \ei
        \vspace{10pt}
        \item Now we get the sequence of strings:
            \bi
                \item $A$
                \item $B$
                \item $AB$
                \item $BAB$
                \item $ABBAB$
                \item $BABABBAB$
                \item $ABBABBABABBAB$
                \item $BABABBABABBABBABABBAB$
                \item $\ldots$
            \ei
    \ei
\end{frame}

\begin{frame}{Fibonacci words}
    \vspace{5pt}
    \bi
        \item How long is $g_n$?
            \bi
                \item $\mathrm{len}(g_1) = 1$
                \item $\mathrm{len}(g_2) = 1$
                \item $\mathrm{len}(g_n) = \mathrm{len}(g_{n-2}) + \mathrm{len}(g_{n-1})$
            \ei
        \vspace{5pt}
        \item Looks familiar?
        \vspace{2pt}
        \item $\mathrm{len}(g_n) = \mathrm{fib}_{n}$
        \vspace{10pt}
        \item So the strings become very large very quickly
            \bi
                \item $\mathrm{len}(g_{10}) = 55$
                \item $\mathrm{len}(g_{100}) = 354224848179261915075$
                \item $\mathrm{len}(g_{1000}) = $ \begin{align*}
                        &434665576869374564356885276750406258025646605173717\\
                        &804024817290895365554179490518904038798400792551692\\
                        &959225930803226347752096896232398733224711616429964\\
                        &409065331879382989696499285160037044761377951668492\\
                        &28875
                    \end{align*}

            \ei
    \ei
\end{frame}

\begin{frame}{Fibonacci words}
    \vspace{40pt}
    \bi
        \item Task: Compute the $i$th character in $g_{n}$
        \vspace{10pt}
        \item<2-> Simple to do in $O(\mathrm{len}(n))$, but that is extremely slow for large $n$
        \vspace{10pt}
        \item<3-> Can be done in $O(n)$ using divide and conquer
    \ei
\end{frame}

\end{document}

