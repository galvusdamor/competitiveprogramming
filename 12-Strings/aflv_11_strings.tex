\documentclass[12pt,t]{beamer}
\subtitle{12 -- Strings}
\newcommand{\shorttitle}{Strings}
\usepackage{graphicx}
\setbeameroption{hide notes}
\setbeamertemplate{note page}[plain]

% get rid of junk
\usetheme{default}
\beamertemplatenavigationsymbolsempty
\hypersetup{pdfpagemode=UseNone} % don't show bookmarks on initial view

% font
\usepackage{fontspec}
\setsansfont{TeX Gyre Heros}
\setbeamerfont{note page}{family*=pplx,size=\footnotesize} % Palatino for notes
% "TeX Gyre Heros can be used as a replacement for Helvetica"
% In Unix, unzip the following into ~/.fonts
% In Mac, unzip it, double-click the .otf files, and install using "FontBook"
%   http://www.gust.org.pl/projects/e-foundry/tex-gyre/heros/qhv2.004otf.zip

% named colors
\definecolor{offwhite}{RGB}{249,242,215}
% \definecolor{foreground}{RGB}{255,255,255}
\definecolor{foreground}{RGB}{0,0,0}
% \definecolor{background}{RGB}{24,24,24}
\definecolor{background}{RGB}{255,255,255}
\definecolor{title}{RGB}{107,174,214}
\definecolor{gray}{RGB}{100,100,100}
\definecolor{subtitle}{RGB}{102,155,204}
\definecolor{hilight}{RGB}{20,180,204}
\definecolor{vhilight}{RGB}{255,111,207}
\definecolor{lolight}{RGB}{155,155,155}
%\definecolor{green}{RGB}{125,250,125}

% use those colors
\setbeamercolor{titlelike}{fg=title}
\setbeamercolor{subtitle}{fg=subtitle}
\setbeamercolor{institute}{fg=gray}
\setbeamercolor{normal text}{fg=foreground,bg=background}
\setbeamercolor{item}{fg=foreground} % color of bullets
\setbeamercolor{subitem}{fg=gray}
\setbeamercolor{itemize/enumerate subbody}{fg=gray}
\setbeamertemplate{itemize subitem}{{\textendash}}
\setbeamerfont{itemize/enumerate subbody}{size=\footnotesize}
\setbeamerfont{itemize/enumerate subitem}{size=\footnotesize}

% settings for table of contents
\setbeamercolor{section in toc}{fg=foreground,bg=background}
\setbeamerfont{subsection in toc}{size=\footnotesize}
\setbeamertemplate{section in toc}{{\scriptsize\leavevmode\raise1.35pt\hbox{$\blacktriangleright$}} \inserttocsection}
%\setbeamertemplate{subsection in toc}{\quad{\tiny\leavevmode\raise1.5pt\hbox{$\blacktriangleright$}} \footnotesize\inserttocsubsection\\}
\setbeamertemplate{subsection in toc}{\quad\quad\textendash\enspace\inserttocsubsection\\}

% page number
\setbeamertemplate{footline}{%
    \raisebox{5pt}{\makebox[\paperwidth]{\hfill\makebox[20pt]{\color{gray}
          \scriptsize\insertframenumber}}}\hspace*{5pt}}

% add a bit of space at the top of the notes page
\addtobeamertemplate{note page}{\setlength{\parskip}{12pt}}

% add subsection as frame title when it is empty
\makeatletter
  \CheckCommand*\beamer@checkframetitle{%
    \@ifnextchar\bgroup\beamer@inlineframetitle{}}
  \renewcommand*\beamer@checkframetitle{%
    \global\let\beamer@frametitle\relax\@ifnextchar%
    \bgroup\beamer@inlineframetitle{}}
\makeatother

\addtobeamertemplate{frametitle}{
  \ifx\insertframetitle\empty
      \frametitle{\insertsubsectionhead}
  \else
  \fi
 }{}


\usepackage{pbox}

% a few macros
\newcommand{\bi}{\begin{itemize}}
\newcommand{\ei}{\end{itemize}}
\newcommand{\ig}{\includegraphics}
\newcommand{\subt}[1]{{\footnotesize \color{subtitle} {#1}}}

% title info
\title{Kompetitives Programmieren}
%\author{Gregor Behnke}
\institute{Prof. Dr. Susanne Biundo-Stephan \\ Gregor Behnke \\ Institute of Artificial Intelligence\\ Ulm University}
\date{\tiny based on Bjarki Ágúst Guðmundsson's and Tómas Ken Magnússon's\\Competitive Programming}


\setbeamertemplate{footline}[text line]{%
  \parbox{\linewidth}{\vspace*{-8pt}\shorttitle \hfill \insertframenumber/\inserttotalframenumber}}
\setbeamertemplate{navigation symbols}{}


\newcommand{\specialcell}[2][c]{%
  \begin{tabular}[#1]{@{}c@{}}#2\end{tabular}}

% Tikz
\usepackage{tikz}
\usepackage{tkz-euclide}
\usetikzlibrary{arrows,shapes}
% \usepackage{intersections}
\usetkzobj{all}
\usetikzlibrary{arrows,shapes,angles,quotes,shapes, calc, decorations,matrix}
\usepackage{forest}
\pgfdeclarelayer{bg}    % declare background layer
\pgfsetlayers{bg,main}


% Minted
\usepackage{minted}
\usemintedstyle{tango}
\newminted{cpp}{fontsize=\footnotesize}

% Graph styles
\tikzstyle{vertex}=[circle,fill=black!50,minimum size=15pt,inner sep=0pt, font=\small]
\tikzstyle{selected vertex} = [vertex, fill=red!24]
\tikzstyle{selected2 vertex} = [vertex, fill=hilight!50, text=black]
\tikzstyle{vertex1} = [vertex, fill=red]
\tikzstyle{vertex2} = [vertex, fill=blue]
\tikzstyle{vertex3} = [vertex, fill=green, text=black]
\tikzstyle{vertex4} = [vertex, fill=yellow, text=black]
\tikzstyle{vertex5} = [vertex, fill=pink, text=black]
\tikzstyle{vertex6} = [vertex, fill=purple]
\tikzstyle{edge} = [draw,thick,-]
\tikzstyle{dedge} = [draw,thick,->]
\tikzstyle{weight} = [font=\scriptsize,pos=0.5]
\tikzstyle{selected edge} = [draw,line width=2pt,-,red!50]
\tikzstyle{ignored edge} = [draw,line width=5pt,-,black!20]


\begin{document}

% title slide
{
    \setbeamertemplate{footline}{} % no page number here
    \frame{
        \titlepage
    }
}





\begin{frame}{Today we're going to cover}
    \vspace{40pt}
    \bi
        \item String matching
            \bi
                \item Naive algorithm
                \item Knuth--Morris--Pratt (KMP) algorithm
            \ei
        \item Tries
        \item Suffix tries
        \item Suffix trees
        \item Suffix arrays
    \ei
\end{frame}

\begin{frame}{String problems}
    \vspace{10pt}
    \bi
        \item Strings frequently appear in our kind of problems
            \bi
                \item Reading input
                \item Writing output
                \item Parsing
                \item Identifiers/names
                \item Data
            \ei
        \vspace{5pt}
        \item But sometimes strings play the key role
            \bi
                \item We want to find properties of some given strings
                \item Is the string a palindrome?
            \ei
        \vspace{5pt}
        \item Here we're going to talk about things related to the latter type of problems
        \vspace{5pt}
        \item These problems can be hard, because the length of the strings are often huge
    \ei
\end{frame}

% TODO: String matching (definition)

\begin{frame}{String matching}
    \vspace{30pt}

	\begin{itemize}
		\item Given a string $S$ of length $n$,
		\item and a string $T$ of length $m$,
		\item find all occurrences of $T$ in $S$
    \end{itemize}
    \begin{itemize}
		\item Note:
        \begin{itemize}
            \item Occurrences may overlap
            \item Assume strings contain characters from a constant-sized alphabet
            % \item Assume $m < n$, otherwise trivial
        \end{itemize}
	\end{itemize}
\end{frame}

\begin{frame}{String matching}
    \vspace{30pt}
	Example:
	\begin{itemize}
		\item $S = \mathrm{cabcababacaba}$
		\item $T = \mathrm{aba}$
	\end{itemize}
    \begin{itemize}
        \item<2-> Three occurrences:
            \begin{itemize}
                \item<3-> $\mathrm{cabc{\color{red}{aba}}bacaba}$
                \item<4-> $\mathrm{cabcab{\color{red}{aba}}caba}$
                \item<5-> $\mathrm{cabcababac{\color{red}{aba}}}$
            \end{itemize}
    \end{itemize}
\end{frame}


% TODO: Naive string matching algorithm

\begin{frame}{Naive string matching algorithm}
    \vspace{30pt}
	\begin{itemize}
	\item For each substring of length $m$ in $S$,
	\item check if that substring is equal to $T$.
	\end{itemize}
		
\end{frame}


\begin{frame}{Naive string matching algorithm}
    \vspace{30pt}
    \begin{itemize}
        \item $S$: \texttt{\textcolor{red}{b}acbababaabcbab}
        \item $T$: \phantom{\texttt{}}\texttt{\textcolor{red}{a}babaca}
    \end{itemize}
\end{frame}
\begin{frame}{Naive string matching algorithm}
    \vspace{30pt}
    \begin{itemize}
        \item $S$: \texttt{b\textcolor{green}{a}\textcolor{red}{c}bababaabcbab}
        \item $T$: \phantom{\texttt{a}}\texttt{\textcolor{green}{a}\textcolor{red}{b}abaca}
    \end{itemize}
\end{frame}
\begin{frame}{Naive string matching algorithm}
    \vspace{30pt}
    \begin{itemize}
        \item $S$: \texttt{ba\textcolor{red}{c}bababaabcbab}
        \item $T$: \phantom{\texttt{aa}}\texttt{\textcolor{red}{a}babaca}
    \end{itemize}
\end{frame}
\begin{frame}{Naive string matching algorithm}
    \vspace{30pt}
    \begin{itemize}
        \item $S$: \texttt{bac\textcolor{red}{b}ababaabcbab}
        \item $T$: \phantom{\texttt{aaa}}\texttt{\textcolor{red}{a}babaca}
    \end{itemize}
\end{frame}
\begin{frame}{Naive string matching algorithm}
    \vspace{30pt}
    \begin{itemize}
        \item $S$: \texttt{bacb\textcolor{green}{ababa}\textcolor{red}{a}bcbab}
        \item $T$: \phantom{\texttt{aaaa}}\texttt{\textcolor{green}{ababa}\textcolor{red}{c}a}
    \end{itemize}
\end{frame}
\begin{frame}{Naive string matching algorithm}
    \vspace{30pt}
    \begin{itemize}
        \item $S$: \texttt{bacba\textcolor{red}{b}abaabcbab}
        \item $T$: \phantom{\texttt{aaaaa}}\texttt{\textcolor{red}{a}babaca}
    \end{itemize}
\end{frame}
\begin{frame}{Naive string matching algorithm}
    \vspace{30pt}
    \begin{itemize}
        \item $S$: \texttt{bacbab\textcolor{green}{aba}\textcolor{red}{a}bcbab}
        \item $T$: \phantom{\texttt{aaaaaa}}\texttt{\textcolor{green}{aba}\textcolor{red}{b}aca}
    \end{itemize}
\end{frame}
\begin{frame}{Naive string matching algorithm}
    \vspace{30pt}
    \begin{itemize}
        \item $S$: \texttt{bacbaba\textcolor{red}{b}aabcbab}
        \item $T$: \phantom{\texttt{aaaaaaa}}\texttt{\textcolor{red}{a}babaca}
    \end{itemize}
\end{frame}
\begin{frame}{Naive string matching algorithm}
    \vspace{30pt}
    \begin{itemize}
        \item $S$: \texttt{bacbabab\textcolor{green}{a}\textcolor{red}{a}bcbab}
        \item $T$: \phantom{\texttt{aaaaaaaa}}\texttt{\textcolor{green}{a}\textcolor{red}{b}abaca}
    \end{itemize}
    % \item<2-> $T$ is always shifted one forward
\end{frame}

\begin{frame}[fragile]{Naive string matching algorithm}
    \begin{minted}[fontsize=\footnotesize]{cpp}
int string_match(const string &s, const string &t) {
    int n = s.size(),
        m = t.size();

    for (int i = 0; i + m - 1 < n; i++) {
        bool found = true;
        for (int j = 0; j < m; j++) {
            if (s[i + j] != t[j]) {
                found = false;
                break;
            }
        }
        if (found) {
            return i;
        }
    }

    return -1;
}
    \end{minted}
\end{frame}

\begin{frame}{Naive string matching algorithm}
    \vspace{30pt}
    \bi
        \item Double for-loop
            \begin{itemize}
                \item outer loop is $O(n)$ iterations
                \item inner loop is $O(m)$ iterations worst case
            \end{itemize}
        \item Time complexity is $O(nm)$ worst case
        \item<2-> Can we do better?
    \ei
\end{frame}

% TODO: KMP algorithm
\begin{frame}
    \frametitle{Knuth--Morris--Pratt algorithm}
    \vspace{30pt}
    \begin{itemize}
    \item The KMP algorithm avoids useless comparisons:
        \begin{itemize}
            \item $S$: \texttt{\textcolor{red}{b}acbababaabcbab}
            \item $T$: \phantom{\texttt{}}\texttt{\textcolor{red}{a}babaca}
        \end{itemize}
    \end{itemize}
\end{frame}
\begin{frame}
    \frametitle{Knuth--Morris--Pratt algorithm}
    \vspace{30pt}
    \begin{itemize}
    \item The KMP algorithm avoids useless comparisons:
        \begin{itemize}
            \item $S$: \texttt{b\textcolor{green}{a}\textcolor{red}{c}bababaabcbab}
            \item $T$: \phantom{\texttt{a}}\texttt{\textcolor{green}{a}\textcolor{red}{b}abaca}
        \end{itemize}
    \end{itemize}
\end{frame}
\begin{frame}
    \frametitle{Knuth--Morris--Pratt algorithm}
    \vspace{30pt}
    \begin{itemize}
    \item The KMP algorithm avoids useless comparisons:
        \begin{itemize}
            \item $S$: \texttt{ba\textcolor{red}{c}bababaabcbab}
            \item $T$: \phantom{\texttt{aa}}\texttt{\textcolor{red}{a}babaca}
        \end{itemize}
    \end{itemize}
\end{frame}
\begin{frame}
    \frametitle{Knuth--Morris--Pratt algorithm}
    \vspace{30pt}
    \begin{itemize}
    \item The KMP algorithm avoids useless comparisons:
        \begin{itemize}
            \item $S$: \texttt{bac\textcolor{red}{b}ababaabcbab}
            \item $T$: \phantom{\texttt{aaa}}\texttt{\textcolor{red}{a}babaca}
        \end{itemize}
    \end{itemize}
\end{frame}
\begin{frame}
    \frametitle{Knuth--Morris--Pratt algorithm}
    \vspace{30pt}
    \begin{itemize}
    \item The KMP algorithm avoids useless comparisons:
        \begin{itemize}
            \item $S$: \texttt{bacb\textcolor{green}{ababa}\textcolor{red}{a}bcbab}
            \item $T$: \phantom{\texttt{aaaa}}\texttt{\textcolor{green}{ababa}\textcolor{red}{c}a}
        \end{itemize}
    \end{itemize}
\end{frame}
\begin{frame}
    \frametitle{Knuth--Morris--Pratt algorithm}
    \vspace{30pt}
    \begin{itemize}
    \item The KMP algorithm avoids useless comparisons:
        \begin{itemize}
            \item $S$: \texttt{bacbab\textcolor{green}{aba}\textcolor{red}{a}bcbab}
            \item $T$: \phantom{\texttt{aaaaaa}}\texttt{\textcolor{green}{aba}\textcolor{red}{b}aca}
        \end{itemize}
    \end{itemize}
\end{frame}
\begin{frame}
    \frametitle{Knuth--Morris--Pratt algorithm}
    \vspace{30pt}
    \begin{itemize}
    \item The KMP algorithm avoids useless comparisons:
        \begin{itemize}
            \item $S$: \texttt{bacbabab\textcolor{green}{a}\textcolor{red}{a}bcbab}
            \item $T$: \phantom{\texttt{aaaaaaaa}}\texttt{\textcolor{green}{a}\textcolor{red}{b}abaca}
        \end{itemize}
    \item<2-> The number of shifts depend on which characters are currently matched
    \end{itemize}
\end{frame}

\begin{frame}
    \frametitle{Knuth--Morris--Pratt algorithm}
    \vspace{10pt}
    \begin{itemize}
        \item How are the number of shifts determined?
            \vspace{5pt}
        \item Let {\footnotesize $\pi[q] = \max \{ k : k < q \textrm{ and } T[1\ldots k] \textrm{ is a suffix of } T[1\ldots q] \}$}
            \vspace{5pt}
        \item<2-> Example:\\
            \begin{center}
                % \hspace{-50px}
                \begin{tabular}{cccccccc}
                    $i$ & $1$ & $2$ & $3$ & $4$ & $5$ & $6$ & $7$ \\
                    \hline
                    $T[i]$&\texttt{a}&\texttt{b}&\texttt{a}&\texttt{b}&\texttt{a}&\texttt{c}&\texttt{a} \\
                    $\pi[i]$ & $0$ & $0$ & $1$ & $2$ & $3$ & $0$ & $1$\\
                \end{tabular}
            \end{center}
    \vspace{5pt}
        \item<3-> If, at position $i$, $q$ characters match (i.e. $T[1\ldots q] = S[i\ldots i+q-1]$), then
            \begin{itemize}
                \item if $q = 0$, shift pattern $1$ position right
                \item otherwise, shift pattern $q - \pi[q]$ positions right
            \end{itemize}
    \end{itemize}
\end{frame}

\begin{frame}
    \frametitle{Knuth--Morris--Pratt algorithm}
    \vspace{30pt}
    \begin{itemize}
        \item Example:
        \begin{itemize}
            \item $S$: \texttt{bacb\textcolor{green}{ababa}\textcolor{red}{a}bcbab}
            \item $T$: \phantom{\texttt{aaaa}}\texttt{\textcolor{green}{ababa}\textcolor{red}{c}a}
            \item<2-> $5$ characters match, so $q = 5$
            \item<3-> $\pi[q] = \pi[5] = 3$
            \item<4-> Then shift $q - \pi[q] = 5 - 3 = 2$ positions
            \item<5-> $S$: \texttt{bacbab\textcolor{green}{aba}\textcolor{red}{a}bcbab}
            \item<5-> $T$: \phantom{\texttt{aaaaaa}}\texttt{\textcolor{green}{aba}\textcolor{red}{b}aca}
        \end{itemize}
    \end{itemize}
\end{frame}

\begin{frame}
    \frametitle{Knuth--Morris--Pratt algorithm}
    \vspace{40pt}
    \begin{itemize}
        \item Given $\pi$, matching only takes $O(n)$ time
        \item $\pi$ can be computed in $O(m)$ time
        \item Total time complexity of KMP therefore $O(n+m)$ worst case
    \end{itemize}
\end{frame}

\begin{frame}[fragile]{Knuth--Morris--Pratt algorithm}
    \begin{minted}[fontsize=\scriptsize]{cpp}
int* compute_pi(const string &t) {

    int m = t.size();
    int *pi = new int[m + 1];
    if (0 <= m) pi[0] = 0;
    if (1 <= m) pi[1] = 0;
    for (int i = 2; i <= m; i++) {
        for (int j = pi[i - 1]; ; j = pi[j]) {
            if (t[j] == t[i - 1]) {
                pi[i] = j + 1;
                break;
            }
            if (j == 0) {
                pi[i] = 0;
                break;
            }
        }
    }

    return pi;
}
    \end{minted}
\end{frame}

\begin{frame}[fragile]{Knuth--Morris--Pratt algorithm}
    \begin{minted}[fontsize=\scriptsize]{cpp}
int string_match(const string &s, const string &t) {

    int n = s.size(),
        m = t.size();

    int *pi = compute_pi(t);

    for (int i = 0, j = 0; i < n; ) {
        if (s[i] == t[j]) {
            i++; j++;
            if (j == m) {
                return i - m;
            }
        }
        else if (j > 0) j = pi[j];
        else i++;
    }

    delete[] pi;
    return -1;
}
    \end{minted}
\end{frame}

\begin{frame}{Sets of strings}
    \vspace{30pt}
    \bi
\item We often have sets (or maps) of strings
\item Insertions and lookups usually guarantee $O(\log n)$ comparisons
    \vspace{10pt}
\item But string comparisions are actually pretty expensive...
    \vspace{10pt}
\item There are other data structures, like tries, which do this in a more clever way
    \ei
\end{frame}

% TODO: Trie
\begin{frame}[fragile]{Tries}

\begin{forest}
for tree={
    circle,
    fill=black!50,
    minimum size=8pt,inner sep=0pt, font=\tiny
  % circle,
  % black,
  % draw,
  % fill=blue!40,
}
  [{}
    [{}, edge label={node [midway, above left] {B}}
      [{}, edge label={node [midway, right] {A}}
        [{}, edge label={node [midway, right] {U}}
          [,phantom]
          [{}, edge label={node [midway, left] {E}}
            [{}, edge label={node [midway, left] {R}}, fill=black!10
            ]
            [,phantom]
          ]
          [{}, edge label={node [midway, right] {M}}, fill=black!10
          ]
          [,phantom]
        ]
        [,phantom]
      ]
      [,phantom]
    ]
    [{}, edge label={node [midway, right] {F}}
      [{}, edge label={node [midway, right] {E}}
        [{}, edge label={node [midway, right] {L}}
          [{}, edge label={node [midway, right] {D}}, fill=black!10
          ]
          [,phantom]
        ]
        [,phantom]
      ]
      [,phantom]
    ]
    [{}, edge label={node [midway, left] {H}}
      [{}, edge label={node [midway, left] {A}}
        [{}, edge label={node [midway, left] {H}}
          [{}, edge label={node [midway, left] {N}}, fill=black!10
          ]
          [,phantom]
        ]
        [,phantom]
        % [{}, edge label={node [midway, right] {U}}
        %   [{}, edge label={node [midway, left] {S}}, fill=black!10
        %   ]
        %   [,phantom]
        % ]
        [,phantom]
      ]
      [{}, edge label={node [midway, left] {O}}
        [{}, edge label={node [midway, right] {F}}, fill=black!10
        ]
      ]
      [{}, edge label={node [midway, right] {U}}
        [,phantom]
        [{}, edge label={node [midway, left] {H}}
          [{}, edge label={node [midway, right] {N}}, fill=black!10
          ]
        ]
        [{}, edge label={node [midway, right] {N}}
          [,phantom]
          [{}, edge label={node [midway, right] {D}}, fill=black!10
          ]
        ]
      ]
    ]
    [{}, edge label={node [midway, above right] {K}}
      [,phantom]
      [{}, edge label={node [midway, right] {A}}
        [,phantom]
        [{}, edge label={node [midway, right] {T}}
          [,phantom]
          [{}, edge label={node [midway, right] {Z}}
            [,phantom]
            [{}, edge label={node [midway, right] {E}}, fill=black!10
            ]
          ]
        ]
      ]
    ]
  ]
\end{forest}
\end{frame}


\begin{frame}[fragile]{Tries}

\begin{forest}
for tree={
    circle,
    fill=black!50,
    minimum size=8pt,inner sep=0pt, font=\tiny
  % circle,
  % black,
  % draw,
  % fill=blue!40,
}
  [{}
    [{}, edge label={node [midway, above left] {B}}
      [{}, edge label={node [midway, right] {A}}
        [{}, edge label={node [midway, right] {U}}
          [,phantom]
          [{}, edge label={node [midway, left] {E}}
            [{}, edge label={node [midway, left] {R}}, fill=black!10
            ]
            [,phantom]
          ]
          [{}, edge label={node [midway, right] {M}}, fill=black!10
          ]
          [,phantom]
        ]
        [,phantom]
      ]
      [,phantom]
    ]
    [{}, edge label={node [midway, right] {F}}
      [{}, edge label={node [midway, right] {E}}
        [{}, edge label={node [midway, right] {L}}
          [{}, edge label={node [midway, right] {D}}, fill=black!10
          ]
          [,phantom]
        ]
        [,phantom]
      ]
      [,phantom]
    ]
    [{}, edge label={node [midway, left] {H}}
      [{}, edge label={node [midway, left] {A}}
        [{}, edge label={node [midway, left] {H}}
          [{}, edge label={node [midway, left] {N}}, fill=black!10
          ]
          [,phantom]
        ]
        % [,phantom]
        [{}, edge label={node [midway, right] {U}}
          [{}, edge label={node [midway, left] {S}}, fill=black!10
          ]
          [,phantom]
        ]
        [,phantom]
      ]
      [{}, edge label={node [midway, left] {O}}
        [{}, edge label={node [midway, right] {F}}, fill=black!10
        ]
      ]
      [{}, edge label={node [midway, right] {U}}
        [,phantom]
        [{}, edge label={node [midway, left] {H}}
          [{}, edge label={node [midway, right] {N}}, fill=black!10
          ]
        ]
        [{}, edge label={node [midway, right] {N}}
          [,phantom]
          [{}, edge label={node [midway, right] {D}}, fill=black!10
          ]
        ]
      ]
    ]
    [{}, edge label={node [midway, above right] {K}}
      [,phantom]
      [{}, edge label={node [midway, right] {A}}
        [,phantom]
        [{}, edge label={node [midway, right] {T}}
          [,phantom]
          [{}, edge label={node [midway, right] {Z}}
            [,phantom]
            [{}, edge label={node [midway, right] {E}}, fill=black!10
            ]
          ]
        ]
      ]
    ]
  ]
\end{forest}
\end{frame}


\begin{frame}[fragile]{Tries}

\begin{forest}
for tree={
    circle,
    fill=black!50,
    minimum size=8pt,inner sep=0pt, font=\tiny
  % circle,
  % black,
  % draw,
  % fill=blue!40,
}
  [{}
    [{}, edge label={node [midway, above left] {B}}
      [{}, edge label={node [midway, right] {A}}
        [{}, edge label={node [midway, right] {U}}
          [,phantom]
          [{}, edge label={node [midway, left] {E}}
            [{}, edge label={node [midway, left] {R}}, fill=black!10
            ]
            [,phantom]
          ]
          [{}, edge label={node [midway, right] {M}}, fill=black!10
          ]
          [,phantom]
        ]
        [,phantom]
      ]
      [,phantom]
    ]
    [{}, edge label={node [midway, right] {F}}
      [{}, edge label={node [midway, right] {E}}
        [{}, edge label={node [midway, right] {L}}
          [{}, edge label={node [midway, right] {D}}, fill=black!10
          ]
          [,phantom]
        ]
        [,phantom]
      ]
      [,phantom]
    ]
    [{}, edge label={node [midway, left] {H}}
      [{}, edge label={node [midway, left] {A}}, fill=black!10
        [{}, edge label={node [midway, left] {H}}
          [{}, edge label={node [midway, left] {N}}, fill=black!10
          ]
          [,phantom]
        ]
        % [,phantom]
        [{}, edge label={node [midway, right] {U}}
          [{}, edge label={node [midway, left] {S}}, fill=black!10
          ]
          [,phantom]
        ]
        [,phantom]
      ]
      [{}, edge label={node [midway, left] {O}}
        [{}, edge label={node [midway, right] {F}}, fill=black!10
        ]
      ]
      [{}, edge label={node [midway, right] {U}}
        [,phantom]
        [{}, edge label={node [midway, left] {H}}
          [{}, edge label={node [midway, right] {N}}, fill=black!10
          ]
        ]
        [{}, edge label={node [midway, right] {N}}
          [,phantom]
          [{}, edge label={node [midway, right] {D}}, fill=black!10
          ]
        ]
      ]
    ]
    [{}, edge label={node [midway, above right] {K}}
      [,phantom]
      [{}, edge label={node [midway, right] {A}}
        [,phantom]
        [{}, edge label={node [midway, right] {T}}
          [,phantom]
          [{}, edge label={node [midway, right] {Z}}
            [,phantom]
            [{}, edge label={node [midway, right] {E}}, fill=black!10
            ]
          ]
        ]
      ]
    ]
  ]
\end{forest}
\end{frame}

\begin{frame}[fragile]{Tries}
    \begin{minted}{cpp}
struct node {
    node* children[26];
    bool is_end;

    node() {
        memset(children, 0, sizeof(children));
        is_end = false;
    }
};
    \end{minted}
\end{frame}

\begin{frame}[fragile]{Tries}
    \begin{minted}{cpp}
void insert(node* nd, char *s) {
    if (*s) {
        if (!nd->children[*s - 'a'])
            nd->children[*s - 'a'] = new node();

        insert(nd->children[*s - 'a'], s + 1);
    } else {
        nd->is_end = true;
    }
}
    \end{minted}
\end{frame}

\begin{frame}[fragile]{Tries}
    \begin{minted}{cpp}
bool contains(node* nd, char *s) {
    if (*s) {
        if (!nd->children[*s - 'a'])
            return false;

        return contains(nd->children[*s - 'a'], s + 1);
    } else {
        return nd->is_end;
    }
}
    \end{minted}
\end{frame}

\begin{frame}[fragile]{Tries}
    \begin{minted}{cpp}
node *trie = new node();

insert(trie, "banani");

if (contains(trie, "banani")) {
    // ...
}
    \end{minted}
\end{frame}

\begin{frame}{Tries}
    \vspace{30pt}
    \bi
        \item Time complexity?
        \vspace{10pt}
        \item Let $k$ be the length of the string we're inserting/looking for
        \item Lookup and insertion are both $O(k)$
        \vspace{10pt}
        \item Also very space efficient...
    \ei
\end{frame}

% TODO: Suffix trie
\begin{frame}{Suffix tries}
    \vspace{20pt}
    \bi
        \item Say we're dealing with some string $S$ of length $n$
        \vspace{10pt}
        \item Let's insert all suffixes of $S$ into a trie
        \vspace{10pt}
    \item $S$ = \texttt{banani}
        \bi
    \item \texttt{insert(trie, "banani");}
    \item \texttt{insert(trie, "anani");}
    \item \texttt{insert(trie, "nani");}
    \item \texttt{insert(trie, "ani");}
    \item \texttt{insert(trie, "ni");}
    \item \texttt{insert(trie, "i");}
        \ei
    \ei
\end{frame}


\begin{frame}[fragile]{Suffix tries}

    \begin{center}
\begin{forest}
for tree={
    circle,
    fill=black!50,
    minimum size=12pt,inner sep=0pt, font=\tiny
  % circle,
  % black,
  % draw,
  % fill=blue!40,
}
  [{}
    [{}, edge label={node [midway, above left] {B}}
      [{}, edge label={node [midway, right] {A}}
        [{}, edge label={node [midway, right] {N}}
          [,phantom]
          [{}, edge label={node [midway, left] {A}}
            [{}, edge label={node [midway, left] {N}}
                [{}, edge label={node [midway, left] {I}}, fill=black!10
                ]
                [,phantom]
            ]
            [,phantom]
          ]
          [,phantom]
        ]
        [,phantom]
      ]
      [,phantom]
    ]
    [{}, edge label={node [midway, below right] {I}}, fill=black!10
        [,phantom]
    ]
    [{}, edge label={node [midway, right] {N}}
        % [,phantom]
        [{}, edge label={node [midway, left] {I}}, fill=black!10
        ]
      [{}, edge label={node [midway, right] {A}}
        [,phantom]
        [{}, edge label={node [midway, right] {N}}
          [,phantom]
          [{}, edge label={node [midway, right] {I}}, fill=black!10
          ]
        ]
      ]
        [,phantom]
    ]
    [{}, edge label={node [midway, above right] {A}}
      [,phantom]
      [{}, edge label={node [midway, right] {N}}
          [{}, edge label={node [midway, left] {I}}, fill=black!10 ]
        [{}, edge label={node [midway, right] {A}}
          [,phantom]
          [{}, edge label={node [midway, right] {N}}
            [,phantom]
            [{}, edge label={node [midway, right] {I}}, fill=black!10
            ]
          ]
        ]
      ]
    ]
  ]
\end{forest}
\end{center}
\end{frame}

% TODO: Applications of Suffix trie
\begin{frame}{Suffix tries}
    \vspace{20pt}
    \bi
        \item There are a lot of cool things we can do with suffix tries
        \vspace{10pt}
        \item Example: String matching
        \vspace{5pt}
        \item If a string $T$ is a substring in $S$, then (obviously) it has to start at some suffix of $S$
        \item So we can simply look for $T$ in the suffix trie of $S$, ignoring whether the last node is an end node or not
        \vspace{5pt}
        \item This is just $O(m)$...
    \ei
\end{frame}

\begin{frame}[fragile]{Suffix tries}

    \begin{center}
\begin{forest}
for tree={
    circle,
    fill=black!50,
    minimum size=12pt,inner sep=0pt, font=\tiny
  % circle,
  % black,
  % draw,
  % fill=blue!40,
}
  [{}, fill=red!50
    [{}, edge label={node [midway, above left] {B}}
      [{}, edge label={node [midway, right] {A}}
        [{}, edge label={node [midway, right] {N}}
          [,phantom]
          [{}, edge label={node [midway, left] {A}}
            [{}, edge label={node [midway, left] {N}}
                [{}, edge label={node [midway, left] {I}}, fill=black!10
                ]
                [,phantom]
            ]
            [,phantom]
          ]
          [,phantom]
        ]
        [,phantom]
      ]
      [,phantom]
    ]
    [{}, edge label={node [midway, below right] {I}}, fill=black!10
        [,phantom]
    ]
    [{}, edge label={node [midway, right] {N}}, fill=red!50
        % [,phantom]
        [{}, edge label={node [midway, left] {I}}, fill=black!10
        ]
      [{}, edge label={node [midway, right] {A}}, fill=red!50
        [,phantom]
        [{}, edge label={node [midway, right] {N}}, fill=red!50
          [,phantom]
          [{}, edge label={node [midway, right] {I}}, fill=black!10
          ]
        ]
      ]
        [,phantom]
    ]
    [{}, edge label={node [midway, above right] {A}}
      [,phantom]
      [{}, edge label={node [midway, right] {N}}
          [{}, edge label={node [midway, left] {I}}, fill=black!10 ]
        [{}, edge label={node [midway, right] {A}}
          [,phantom]
          [{}, edge label={node [midway, right] {N}}
            [,phantom]
            [{}, edge label={node [midway, right] {I}}, fill=black!10
            ]
          ]
        ]
      ]
    ]
  ]
\end{forest}
\end{center}
\end{frame}

% TODO: Construction of Suffix tries is too slow...
\begin{frame}{Suffix tries}
    \vspace{20pt}
    \bi
        \item String matching is fast if we have the suffix trie for $S$
            \vspace{10pt}
        \item But what is the time complexity of suffix trie construction?
        \item There are $n$ suffixes, and it takes $O(n)$ to insert each of them
        \item So $O(n^2)$, which is pretty slow
            \vspace{10pt}
        \item Can we do better?
        \item There can be up to $n^2$ nodes in the graph, so this is actually optimal...
    \ei
\end{frame}

% TODO: Suffix tree
\begin{frame}{Suffix trees}
    \vspace{30pt}
    \bi
        \item There exists a compressed version of a suffix trie, called a suffix tree
        \vspace{10pt}
    \item It can be constructed in $O(n)$, and has all the features that suffix tries have
    \item But the $O(n)$ construction algorithm is pretty complex, a big disadvantage for us
    \ei
\end{frame}

% TODO: Suffix array
\begin{frame}{Suffix arrays}
    \vspace{30pt}
    \bi
        \item A variation of the previous structures
        \item Can do everything the other structures can do, with a small overhead
        \item Can be constructed pretty quickly with relatively simple code
    \ei
\end{frame}

\begin{frame}[fragile]{Suffix arrays}
    \bi
        \item Take all the suffixes of $S$
    \ei
    \begin{minted}{cpp}
banani
anani
nani
ani
ni
i

    \end{minted}
    \bi
        \item and sort them
    \ei
    \begin{minted}{cpp}

anani
ani
banani
i
nani
ni
    \end{minted}

\end{frame}

\begin{frame}[fragile]{Suffix arrays}
    \vspace{40pt}

    \bi
        \item We can use this array to do everything that suffix tries can do
            \vspace{10pt}
        \item Like string matching
    \ei
\end{frame}

\begin{frame}[fragile]{Suffix arrays}
    \vspace{10pt}
    \bi
\item Let's look for \texttt{nan}
\item<2-> The first letter in the string has to be \texttt{n}, so we can binary search for the range of strings starting with \texttt{n}
    \ei
    \vspace{10pt}
    \begin{minted}{cpp}

anani
ani
banani
i
nani
ni
    \end{minted}
\end{frame}

\begin{frame}[fragile]{Suffix arrays}
    \vspace{10pt}
    \bi
\item Let's look for \texttt{nan}
\item The first letter in the string has to be \texttt{n}, so we can binary search for the range of strings starting with \texttt{n}
    \ei
    \vspace{10pt}
    \begin{minted}{cpp}





nani
ni
    \end{minted}
\end{frame}

\begin{frame}[fragile]{Suffix arrays}
    \vspace{10pt}
    \bi
\item Let's look for \texttt{nan}
\item The second letter in the string has to be \texttt{a}, so we can binary search for the range of strings that have \texttt{a} as the second letter
    \ei
    \vspace{10pt}
    \begin{minted}{cpp}





nani
ni
    \end{minted}
\end{frame}

\begin{frame}[fragile]{Suffix arrays}
    \vspace{10pt}
    \bi
\item Let's look for \texttt{nan}
\item The second letter in the string has to be \texttt{a}, so we can binary search for the range of strings that have \texttt{a} as the second letter
    \ei
    \vspace{10pt}
    \begin{minted}{cpp}
nani
    \end{minted}
\end{frame}

\begin{frame}[fragile]{Suffix arrays}
    \vspace{10pt}
    \bi
\item Let's look for \texttt{nan}
\item The third letter in the string has to be \texttt{n}, so we can binary search for the range of strings that have \texttt{n} as the third letter
    \ei
    \vspace{10pt}
    \begin{minted}{cpp}
nani
    \end{minted}
\end{frame}

\begin{frame}[fragile]{Suffix arrays}
    \vspace{10pt}
    \bi
\item Let's look for \texttt{nan}
\item The third letter in the string has to be \texttt{n}, so we can binary search for the range of strings that have \texttt{n} as the third letter
    \ei
    \vspace{10pt}
    \begin{minted}{cpp}
nani
    \end{minted}
    \bi
\item<2-> If there is at least one string left, we have a match
    \ei
\end{frame}

\begin{frame}{Suffix arrays}
    \vspace{30pt}
    \bi
        \item Time complexity?
            \vspace{5pt}
        \item For each letter in $T$, we do two binary searches on the $n$ suffixes to find the new range
        \item Time complexity is $O(m \times \log n)$
            \vspace{10pt}
        \item A bit slower than doing it with a suffix trie, but still not bad
    \ei
\end{frame}

\begin{frame}{Suffix arrays}
    \vspace{30pt}
    \bi
        \item But how do we construct a suffix array for a string?
            \vspace{10pt}
        \item A simple \texttt{sort(suffixes)} is $O(n^2\log(n))$, because comparing two suffixes is $O(n)$
        \item And we still have the same problem as with suffix tries, there are almost $n^2$ characters if we store all suffixes
    \ei
\end{frame}

\begin{frame}[fragile]{Suffix arrays}
    \bi
        \item The second problem is easy to fix
        \item Just store the indices of the suffixes
    \ei
    \begin{minted}[fontsize=\footnotesize]{cpp}
anani
ani
banani
i
nani
ni
    \end{minted}
    \bi
        \item becomes
    \ei
    \begin{minted}[fontsize=\footnotesize]{cpp}
1: anani
3: ani
0: banani
5: i
2: nani
4: ni
    \end{minted}
\end{frame}

\begin{frame}{Suffix arrays}
    \bi
        \item What about the construction?
        \item In short, we
            \bi
                \item sort all suffixes by only looking at the first letter
                \item sort all suffixes by only looking at the first 2 letters
                \item sort all suffixes by only looking at the first 4 letters
                \item sort all suffixes by only looking at the first 8 letters
                \item \ldots
                \item sort all suffixes by only looking at the first $2^i$ letters
                \item \ldots
            \ei
        \vspace{10pt}
    \item If we use an $O(n\log n)$ sorting algorithm, this is $O(n\log^2 n)$
    \item We can also use an $O(n)$ sorting algorithm, since all sorted values are between $0$ and $n$, bringing it down to $O(n \log n)$
    \ei
\end{frame}

\begin{frame}[fragile]{Suffix arrays}
    \begin{minted}[fontsize=\footnotesize]{cpp}
struct suffix_array {
    struct entry {
        pair<int, int> nr;
        int p;

        bool operator <(const entry &other) {
            return nr < other.nr;
        }
    };


    string s;
    int n;
    vector<vector<int> > P;
    vector<entry> L;
    vi idx;

    // constructor
};
\end{minted}
\end{frame}

\begin{frame}[fragile]{Suffix arrays}
    \begin{minted}[fontsize=\tiny]{cpp}
suffix_array(string _s) : s(_s), n(s.size()) {
    L = vector<entry>(n);
    P.push_back(vi(n));
    idx = vi(n);

    for (int i = 0; i < n; i++) {
        P[0][i] = s[i];
    }

    for (int stp = 1, cnt = 1; (cnt >> 1) < n; stp++, cnt <<= 1) {
        P.push_back(vi(n));
        for (int i = 0; i < n; i++) {
            L[i].p = i;
            L[i].nr = make_pair(P[stp - 1][i], i + cnt < n ? P[stp - 1][i + cnt] : -1);
        }

        sort(L.begin(), L.end());
        for (int i = 0; i < n; i++) {
            if (i > 0 && L[i].nr == L[i - 1].nr) {
                P[stp][L[i].p] = P[stp][L[i - 1].p];
            } else {
                P[stp][L[i].p] = i;
            }
        }
    }

    for (int i = 0; i < n; i++) {
        idx[P[P.size() - 1][i]] = i;
    }
}
\end{minted}
\end{frame}

\begin{frame}[fragile]{Suffix arrays}
    \bi
        \item There is also one other useful operation on suffix arrays
        \item Finding the longest common prefix (lcp) of two suffixes of $S$
    \ei

    \begin{minted}[fontsize=\footnotesize]{cpp}
1: anani
3: ani
0: banani
5: i
2: nani
4: ni
    \end{minted}

    \bi
\item \texttt{lcp(1,3) = 2}
\item \texttt{lcp(2,1) = 0}
    \vspace{10pt}
\item This function can be implemented in $O(\log n)$ by using intermediate results from the suffix array construction
    \ei
\end{frame}

\begin{frame}[fragile]{Suffix arrays}
    \begin{minted}[fontsize=\footnotesize]{cpp}
int lcp(int x, int y) {
    int res = 0;
    if (x == y) return n - x;
    for (int k = P.size() - 1; k >= 0 && x < n && y < n; k--) {
        if (P[k][x] == P[k][y]) {
            x += 1 << k;
            y += 1 << k;
            res += 1 << k;
        }
    }
    return res;
}
\end{minted}
\end{frame}

\begin{frame}{Longest common substring}
    \vspace{30pt}
    \bi
        \item Given two strings $S$ and $T$, find their longest common substring
            \vspace{10pt}
        \item \texttt{S = banani}
        \item \texttt{T = kanina}
            \vspace{10pt}
        \item Their longest common substring is \texttt{ani}
            \vspace{20pt}
        \item \textit{see example}
    \ei
\end{frame}

\end{document}

